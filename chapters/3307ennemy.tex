\section{Ennemi}

Lors de la planification, il faut tenir compte de la disposition des forces ennemies, leur composition, son \gls{aob}, et ses intentions probables.

\begin{e1}
	
	\item Autres considérations:
	
	\begin{e2}
		
		\item Quelles sont les capacités offensives et défensives de l'ennemi?
		
		\item Quelles sont les capacités ennemies concernant:
		
		\begin{e3}
			\item Menaces sol-air.
			\item Leurres.
			\item Camouflage.
		\end{e3}
		
		Les cibles de hautes importance seront généralement défendue par des unités anti-aériennes. L'utilisation d'armes à très longue portée et le fait de varier les \glspl{ip} augmentera les chances de survies des appareils de \gls{cas}.
		
		\item Quelles sont les capacités de l'ennemi en terme d'\gls{ew} ou de \gls{cc}?
		
	\end{e2}
	
	\item A partir de ces informations, le planificateur peut anticiper les capacités de l'ennemi à contrer la mission alliée, et l'influence potentielle qu'aura l'ennemi sur les vols alliés.
	
	Les changements susceptibles de se produire pendant le déroulement de la mission rendent les communications avec les appareils en vol très importantes, nécessitant parfois une adaptation du plan de vol, c'est pourquoi des alternatives sont prévues dès le commencement de la mission.
\end{e1}