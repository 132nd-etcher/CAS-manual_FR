\section{Planification TAC}

\subsection{Introduction}

L'autorité du \gls{taca} peut varier de simple relais radio jusqu'au contrôle des appareils et du tir.

Pour être efficace, le \gls{taca} doit planifier sa mission en détail et de façon intégrée.

Le \gls{taca} doit être familier avec le même ensemble de documents que le \gls{faca} (\fullref{facaplanning}).

Les responsabilités possibles du \gls{taca} incluent:

\begin{e1}
	\item Coordonner le support aérien offensif:
	\begin{e2}
		\item Fournir le CAS-brief et les \glspl{tot}.
		\item Transférer les appareils aux autres contrôleurs.
		\item Faire office de relais pour les mises à jour de la situation et les \glspl{bda}.
		\item Coordonner les tirs.
		\item Servir de \gls{faca} temporaire (si qualifié)
	\end{e2}
	\item Coordonner ou exécuter le \gls{cc}:
	\begin{e2}
		\item Augmenter la portée du réseau en servant de relais.
		\item Contrôler une partie de l'espace aérien.
		\item Assumer les responsabilités d'une autre agence \gls{cc} en son absence.
	\end{e2}
	\item Coordonner les opérations de support:
	\begin{e2}
		\item Coordonner les missions d'évacuation ou de sauvetage.
		\item Traiter les demandes de support aérien.
		\item Fournir un support pour les opération héliportées.
		\item Coordonner les unités \gls{sead} on-call.
		\item Coordonner le feu de surface.
	\end{e2}
\end{e1}

\subsection{Planning avant la mission}

Le \gls{taca} est souvent géographiquement éloigné des unités qu'il soutient.

Cela ne doit pas l'empêcher de prendre une part active au processus de planification de la mission, et de tenir à jour les différents documents opérationnels.