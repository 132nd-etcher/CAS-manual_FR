\newpage
\section{Hélicoptères: la 5-Line}

La \gls{5l} se concentre sur les forces alliées et est utilisée pour orienter rapidement les \glspl{rw}.

\efig{5line}{5-Line}

La \gls{5l} est considérée comme un CAS brief au même titre que la \gls{9l}, et \textbf{la transmission de ce briefing n'inclut pas l'autorisation de tir implicite}.

La \gls{5l} peut être utilisée avec des appareils et des pilotes qui n'ont pas l'habitude de travailler avec les procédures \gls{cas} en vigueur, auquel cas le \ja{} devra préciser "Sur mon ordre", pour spécifier que le tir n'est pas libre, et donner un avertissement avant de transmettre la \gls{5l}.

\e
	\item De par sa nature, la \gls{5l} suppose que les \gls{rw} ont une \gls{sa} des alliés suffisante pour les localiser et acquérir la cible en se servant de leur position comme référence. Si cette \gls{sa} n'est pas acquise, un CAS brief centré sur la cible (\gls{9l}) devra être utilisé. Par exemple, pour des \glspl{rw} en \gls{cas} immédiat qui viennent d'arriver sur la \gls{flot}.
	\ee
		\itemt{Ligne 1 - Avertissement}{
		L'avertissement sert à prévenir les pilotes \gls{rw} qu'ils vont recevoir un briefing d'attaque. Cet avertissement doit inclure le game plan, le type de contrôle, la méthode d'attaque et l'armement souhaité (si d'application). Les intervalles ne s'appliquent généralement pas à une \gls{5l}}.
		\item{Ligne 2 - Position des alliés/Position de la marque}{
		Le \gls{jtac} transmet la position des alliés et la façon dont il est marqué.}
		\itemt{Ligne 3 - Position de la cible}{
		Le \gls{jtac} transmet la position de la cible, via l'un des moyens suivants:}
		\eee
			\item Direction et distance à partir du point décrit en Ligne 2
			\item \gls{trp}
			\item Position \gls{grg}
			\item Un offset à partir de l'un de ces trois points
		\ed
		Généralement, comme dans le cas d'une \gls{5l} le pilote est généralement la tête hors du cockpit en train de chercher la cible, la transmission de coordonnées n'est pas la meilleure solution.
		\item Ligne 4 - Description de la cible et type de marque
		\eee
			\item La description de la cible doit être suffisamment précise que pour que le pilote identifier la cible, mais rester brève et concise. Des informations supplémentaires pourront être transmises pendant l'approche.
			\item Si la cible est marquée, le type de marque doit également être communiqué
		\ed
		\item Ligne 5 - Remarques
		\eee
			\item La section des remarques doit inclure toute autre information à la conduite de l'attaque, parmi lesquelles:
			\eeee
				\item \gls{fah}
				\item Menaces
				\item Plan \gls{sead}
				\item Plan d'illumination
				\item Mesures de contrôle de l'espace aérien
				\item Dangers à la navigation
				\item Météo dans la zone
				\item Danger proche
				\item Coordination temporelle
			\ed
			\item L'utilisation de la \gls{5l} suppose que les appareils \gls{cas} se dirigeront immédiatement vers la cible après réception du briefing et le read-back. Le \gls{jtac} peut fournir un \gls{tot}, mais doit clairement le spécifier lors du briefing, de manière à ce que les appareils \gls{cas} ne s'approchent pas prématurément de la zone
		\ed
		\item Un read-back pour une \gls{5l} doit inclure toutes les restrictions
		\efig{5line2}{Exemple de 5-Line avec un pilote qualifié au CAS}

	\ed
\ed