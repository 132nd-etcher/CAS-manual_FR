\chapter{Planification et demande}

\begin{center}
    \makebox[\textwidth]{\includegraphics[width=\paperwidth]{quote3.png}}
\end{center}

\section{Introduction}

Ce chapitre décrit le processus de prise de décision \gls{cas}, décrit les responsabilités du staff, établit les bases de ce qu'il y a lieu de prendre en compte lors de la planification, et établit les procédures de requêtes de \gls{cas}.

La phase de planification commence lorsque l'unité reçoit l'ordre du Haut Commandement.

Bien que chapitre se concentre principalement sur les tâches à effectuer lors d'opération complexes, les même tâches peuvent s'appliquer aux opérations d'exfiltration, de \gls{csar}, qui pourraient avoir une structure de commandement différente.


\e
    \item Il existe deux types de \gls{cas} distincts:
    \ee
        \item Le \gls{cas} planifié
        \item Le \gls{cas} immédiat
    \ed
\ed

\section{Mission}

Le \gls{cas} est intégré au reste du dispositif de la coalition en opérations défensives et offensives, pour détruire, neutraliser, interdire, retarder ou empêcher le mouvement de l'ennemi.
\e
	\item Le \gls{cas} peut servir pour renforcer les opérations annexes, soutenir l'effort majeur ou établir les zones de sécurité
	\ee
		\itemt{Opérations annexes}{
		Bien que ce ne soit pas vocation principale, le \gls{cas} peut être employé pour soutenir une opération loin en territoire ennemie (infiltration, exfiltration, opérations spéciales), ou pour une opération ponctuelle}
		\itemt{Opérations de combat rapproché}{
		Le plus souvent, le \gls{cas} servira à renforcer l'effort de guerre principal. Les appareils de \gls{cas} ajoutent leur force aux forces au sol pour soutenir les \glspl{gc}. La vitesse et la portée du \gls{cas} en font un élément idéal pour exploiter les avancées amies, contrer les manoeuvres adverses ou poursuivre un ennemi en fuite.}
		\itemt{Opérations de sécurité}{
		Le \gls{cas} est efficace pour empêcher la pénétration de l'ennemi. Le temps de réponse et la puissance de feu du \gls{cas} peut augmenter de beaucoup la force des unités au sol.}
	\ed
	\item Le \gls{cas} peut servir pour les opérations offensives, défensives et de stabilité
	\ee
		\itemt{Le \gls{cas} en support de l'offense}{
		\eee
			\itemt{Mouvement vers le contact}{
			Le \gls{cas} peut servir à appuyer les forces au sol qui font mouvement. Une fois le contact avec l'ennemi établi, le \gls{cas} peut les submerger et accélérer le déploiement des forces alliées. \textbf{Pendant la planification de l'intégration du \gls{cas} à l'avancée alliée, il est recommandé de déployer les appareils tout le long de l'axe de progression.}}
			\itemt{Attaque}{
			Les \gls{gc} peuvent utilser le \gls{cas} pour attaquer directement l'ennemi. Le \gls{cas} peut détruire les unités ou capacités ennemies importantes avant que l'ennemi ne puisse établir de défense. Le \gls{cas} peut également servir à renforcer la puissance de feu lors d'une attaque, ou pour isoler une force ennemie sur le champ de bataille et la forcer à prendre une position défensive.}
			\itemt{Exploitation}{
			L'exploitation est une tactique offensive utilisée après une attaque réussie, et sert à déstabiliser l'ennemi, en coupant les voies d'accès, en détruisant les forces qui font retraite, et pour frapper les cibles d'opportunité lorsque la cohésion de l'ennemi diminue.}				\itemt{Poursuite}{
			En poursuite, le \gls{gc} attaque l'efficacité de l'ennemi en fuite alors qu'il est démoralisé et que sa cohésion et sont contrôle sont faibles. Puisque l'objectif de la poursuite est la destruction de l'ennemi, le \gls{cas} peut maintenir une pression directe et constante sur l'ennemi pour empêcher qu'il ne se réorganise ou ne se reconstitue.}
		\ed}
		\item Le \gls{cas} en support des opérations défensives alliées
		\eee
			\itemt{Support de la manoeuvre}{
			Le \gls{cas} peut s'ajouter à la puissance de feu des unités au sol en tant que membre d'une force combinée}
			\itemt{Support du mouvement}{
			Le \gls{cas} peut soutenir les éléments alliés lors de leurs mouvements entre deux positions. Les \gls{gc} peuvent utiliser le \gls{cas} pour protéger le front, les flancs ou l'arrière de la formation en mouvement.}
			\itemt{Refoulement de l'ennemi}{
			Le \gls{cas} peut servir à refouler une force ennemie qui aurait franchi ou pénétré les défenses alliées}
		\ed
		\itemt{Le \gls{cas} en opération de stabilité}{
		L'emploi du \gls{cas} en opération de stabilité est foncièrement différent de l'emploi du \gls{cas} en opération de combat. Comme l'objectif d'une opération de stabilité est d'établir la sécurité civile, rétablir les services, et restaurer et protéger les infrastructures, le \gls{cas} en opération de stabilité sera généralement restreint en terme d'échelle et des \glspl{roe} strictes seront souvent d'application. L'utilisation de \glspl{pgm} sera souvent préconisé par le \gls{jfc} lors du support de \gls{gc} en environnement urbain, de manière à limiter les dommages collatéraux.
		}
	\ed
\ed

\section{Ennemi}

% reprendre page 82

\section{Le CAS planifié}

\e
    \item Le ``\gls{cas} request'' est effectué par l'unité qui demande un soutien aérien. On distingue deux principaux types de demandes:
    \ee
        \item Le \gls{cas} planifié à l'avance (``pre-planned \gls{cas}'', ou simplement ``\gls{cas}'')
        \item Le \gls{cas} ``à la demande'' (``on-call cas''), qui regroupe deux possiblités:
        \eee
            \item Le \gls{xcas}, où l'appareil en attente de tasking se trouve déjà dans les airs
            \item Le \gls{gcas}, où l'appareil en attente de tasking se trouve encore au sol
        \ed
    \ed
    \item Un ``\gls{cas} request'' réel se compose d'un grand nombre d'informations, utiles à la planification de la mission et à la répartition des effectifs disponibles.
    \item
    Souvent, pour DCS, l'équivalent du ``\gls{cas} request'' sera contenu dans le briefing de la mission (zone d'opération, menace, composition et force de unités ennemies, timing, emport, unités amies, \gls{jtac}, plan de fréquences, etc…).
    \item Toute référence ultérieure au ``\gls{cas} request'' fera donc implicitement référence au briefing de mission.
    \item Les considération à prendre en compte lors de l'établissement d'une \gls{cas} request sont propres au rôle de \gls{gc}/\gls{jfc} et ne seront pas traitées dans ce document.
    \item Le \gls{cas} planifié est un \gls{cas} effectué après une requête spécifique, concernant une zone connue, pour atteindre un objectif déterminé.
    \item Dans le cas d'un \gls{cas} planifié, un grand nombre de paramètres sont connus avant le décollage, ce qui permet une meilleure préparation.
\ed

\subsection{Étape 1: réception de l'ordre de mission}

\e
    \item
    En tant que participants au processus de planification, les officiers en charge (chef de patrouille, commandant d'escadron, …) devraient être à même de comprendre et analyser l'information à partir de ces différentes sources:
    \ee
        \item \gls{aob}
        \item \gls{ato}
        \item \gls{aco}
        \item \gls{spins}
        \item \gls{opord}
        \item \gls{sop}
    \ed
\ed
\note{Les \glspl{sop} sont propres à l'escadron, et sont normalement connues par les tous les officiers supérieurs.
    
    Exceptés les \glspl{sop}, tous les éléments sont normalement fournis sous une forme ou l'autre dans le briefing de mission.
    
    S'il devait manquer un élément, il appartient à l'officier en charge de la patrouille d'évaluer son importance et de poser les questions nécessaires à l'organisateur de la mission.
}

\subsection{Étape 2: analyse de la mission} 

\e
    \item Avant de pouvoir préparer la mission, les officiers en charges devront:
    \ee
        \item Mettre à jour les différentes sources d'informations (\gls{ato}, \gls{spins}, \gls{aco}, …)
        \item Estimer les capacités des forces à leur disposition (équipement, personnel, restrictions, …)
        \item Déterminer les tâches essentielles de la mission
        \item Évaluer les conditions relatives à:
        \eee
            \item L'ennemi
            \item La météo
            \item Le terrain
        \ed
        \item Avertir les unités (personnes) subordonnées
    \ed
    \item Considérations clefs à prendre en compte lors de l'analyse de la mission:
    \ee
        \item \gls{conops}: Quelles sont les intentions du \gls{jfc}? Attaque ou de défense? Surprise ou délibéré? Règles d'engagement?
        \item Comment s'intègre le \gls{cas} au reste du dispositif?
        \item Quelles pourraient être les intentions de l'ennemi? Comment ces intentions sont-elles affectées par le terrain, la météo, l'heure ?
        \item Quels sont les moyens de surveillance ou de reconnaissance à notre disposition?
        \item \gls{complan} Comment vont s'organiser les communications? Est-ce que toutes les unités participantes au \gls{cas} sont intégrées au \gls{complan}  de manière fiable et redondante?
    \ed
    \item Dans le cas d'une demande de \gls{cas} pré-planifié, un grand nombre d'informations supplémentaires pourront être obtenues via la demande elle-même. Par exemple:
    \ee
        \item Zone de \gls{cas}
        \item Menaces
        \item Type de cibles
        \item Localisation de la cible
        \item Localisation des forces amies
        \item Type de terrain
        \item Restrictions horaires
        \item \gls{jtac}
    \ed
\ed

\subsection{Étape 3: préparation de la mission}

\e
    \item Le ``Mission planning'', ou ``préparation de la mission'', est une étape cruciale au bon déroulement de cette dernière.
    \item Cette préparation sera effectuée par le chef de patrouille, qui sera chargé de rassembler et analyser toutes les informations à sa disposition pour préparer au mieux la mission de la patrouille dont il a la charge.
    \item Lorsque c'est possible, la préparation de la mission se fait en collaboration avec les différents éléments impliqués, certainement avec le \gls{jtac} en charge de la zone attribuée à la patrouille, le \gls{tacon}, et les autres chefs de patrouilles opérant dans le même théâtre opérationnel.
    \item Le chef de patrouille veillera tout particulièrement à toujours utiliser les informations les plus récentes, et fera attention à se tenir au courant de l'évolution de la situation.
    \item Il devra également s'assurer que tous les éléments sous son contrôle ont reçu et compris ses intentions pour la mission.
    \item Checklist de préparation de mission: \fullref{ann1}
\ed

\section{Le CAS immédiat}

\e
    \item Le \gls{cas} immédiat intervient lorsque la situation sur le champ de bataille évolue de manière inattendue.
    \item L'\gls{ato} peut allouer un certain effectif pour se préparer à répondre à ces situations.
    \item Le chef de patrouille trouvera alors dans cet \gls{ato} les informations nécessaires pour préparer son vol (emport, point d'attente, etc…).
    \item
    Pendant le vol (ou au parking pour le \gls{gcas}), si l'intervention de la patrouille s'avère nécessaire, celle-ci recevra un point de contact initial, ainsi que le call-sign et la fréquence de l'agence qui s'occupera de son contrôle terminal.
\ed





