\section{Planification FAC(A)}

Le \gls{faca} peut servir de contrôleur additionnel pour le \gls{jtac}.

Un \gls{faca} doit être à même de coordonner l'effort des appareils de \gls{cas} sans l'assistance du \gls{jtac}, de jour comme de nuit, quelles que soient les conditions météo. Il doit également pouvoir intégrer le \gls{cas} aux autres éléments du champ de bataille, mitiger le tir fratricide, et exécuter l'intégration et la planification avec les éléments de support.

\begin{e1}
	\item Avant la mission
	\begin{e2}
		\item Pendant la planification de la mission, le \gls{jtac} doit conseiller l'échelon supérieur quant à l'emploi et l'intégration du \gls{cas} et du \gls{faca}. L'officier en charge de la planification Air doit avoir une connaissance fonctionnelle du \gls{cas} et du \gls{faca}, ainsi que de leurs limitations.
		C'est le \gls{jtac} qui effectue la demande pour obtenir un \gls{faca} si:
		\begin{e3}
			\item Il s'attend à avoir un grand nombre d'appareils \gls{cas} dans un temps restreint ou dans un petit espace aérien
			\item Il doit travailler en environnement confiné (ville, forêt) où une plateforme disposant de la même perspective que les appareils de \gls{cas} sera d'une grande aide au talk-on
			\item Il dispose d'une capacité limitée à marquer les cibles
			\item Il s'attend à avoir des difficultés de communications à cause du terrain
			\item Le besoin opérationnel implique d'avoir un pilote familier avec les intentions du \gls{gc} et les procédures \gls{cas} pour assister dans la bataille/opération
		\end{e3}
		\item Pour de grandes opérations, le \gls{jtac} ne doit pas hésiter à demander à ce que le \gls{faca} soit présent lors du processus de planification, pour y apporter son expertise et augmenter par là la \gls{sa} des pilotes et leur efficacité au combat.
		\item Le \gls{faca} n'aide pas seulement de par sa connaissance approfondie de l'appareil et de l'armement, mais également pour la préparation du plan \gls{sead}, la prise en compte de la météo, et quantité d'autre facteurs essentiels à la réussite de la mission.
		\item Si le \gls{faca} ne peut être présent lors de la planification, il appartient au \gls{jtac} de conseiller le \gls{gc} quant à l'emploi du \gls{faca}.
		\item Une intégration détaillée et une coordination avant le vol permettent d'établir un plan commun, à partir duquel des déviations pourront être établies si la situation change. Si cette intégration n'a pas lieu, la mission ne sera pas un échec pour autant mais les éléments seront moins bien préparés et commenceront l'opération avec une \gls{sa} moindre. Si cette planification est effectuée, il ne faut plus transmettre que les mises à jour lors du check-in du \gls{faca}. Les personnes et documents suivants servent de source d'information pour la planification:
		\begin{e3}
			\item \gls{fscoord} [ne s'applique généralement pas à DCS]
			\begin{e4}[-1em]
				\item Plan de support
				\item Cibles de grande importance
				\item Communications
				\item Liste des cibles
				\item Artillerie alliée
				\item \glspl{fscm}
				\item \glspl{sop} pour le \gls{sead}
				\item Plan pour l'emploi du laser
			\end{e4}
			\itemt{\gls{opord}}{
			L'\gls{opord} est une directive émanant d'une officier commandant vers ses officiers subalternes pour coordonner l'exécution d'une opération. Un grand nombre d'informations nécessaire à la planification de la mission peuvent se trouver dans l'\gls{opord}, et ce dernier doit être lu et compris par le planificateur \gls{faca}. Les sections suivantes sont particulièrement importantes:}
			\begin{e4}
				\item Operations
				\begin{e5}
					\itemt{Situation alliée}{
					Statut et mission des unités alliées}
					\itemt{Manoeuvres}{
					Zones opérationnelles, limites de manoeuvre des unités alliées, et lignes de phase}
					\itemt{Effort principal}{
					Là où sera concentré l'effort de guerre pour chaque phase de l'opération}
					\itemt{Unités de reconnaissance}{
					Leur positions initiales et prévues, leur mission, les unités qui les soutiennent, le réseau de communications, leurs capacités à amrquer les cibles, et les moyens par lesquels on peut les identifier comme unités amies}
					\item Position des unités des forces spéciales
					\item Restrictions intrinsèques aux \glspl{roe}
				\end{e5}
				\item Renseignement
				\begin{e5}[-1em]
					\item Priorités
					\item Cibles
					\item \glspl{coa} ennemie possibles et probable
					\item Estimations
					\item Plan de rassemblement des informations
					\item Ordre de bataille des unités ennemies au sol
					\item Ordre de bataille des unités aériennes ennemies
				\end{e5}
				\item Tir de soutien
				\begin{e5}[-1em]
					\item Plan de manoeuvre
					\item Plan de tir de soutien
					\item \glspl{roe}
					\item \gls{cas} déjà planifié (pré-planifié et on-call)
					\item Cibles aériennes
					\item Plan de tir d'artillerie
					\item Cibles de l'artillerie
					\item Positions initiales, capacités de tir et limites de l'artillerie
					\item Plan de tir des unités navales
				\end{e5}
				\item Communications
				\begin{e5}[-1em]
					\item Architecture du réseau
					\item Réseaux prévus
					\item Procédures d'authentification
					\item \gls{comsec}
				\end{e5}
				\item Opération aériennes
				\begin{e5}[-1em]
					\item Procédures de contrôle
					\item Procédures \gls{faca}
					\item Procédure \gls{misrep}
					\item Marquages des cibles pour une attaque aérienne
					\item Missions d'\gls{ai} et de \gls{ar}
					\item CAS briefing
					\item Briefing des hélicoptères d'attaque
					\item Armement
					\item \glspl{acm}
					\item Routing tactique
				\end{e5}
				\itemt{\glspl{sop} du théâtre d'opération}{
				Ces documents apportent un supplément d'informations à l'\gls{opord}.}
				\itemt{\gls{ato}}{
				L'\gls{ato} contient le plan du \gls{jfacc} pour fournir le \gls{cas} requis dans l'\gls{oplan}/\gls{opord}. Le \gls{faca} doit lire attentivement l'\gls{ato}, l'\gls{aco} et les \gls{spins} pour acquérir les informations suivantes:}
				\begin{e5}[-1em]
					\item Unités \gls{cas} et \glspl{faca} disponibles (numéros de mission, \gls{tos}, armement, etc.)
					\item Routing (\gls{fw} et \gls{rw})
					\item Points de contrôles
					\item \glspl{acm}
					\item Zone d'opération prévue
					\item Disponibilité des tankers
					\item Mots-code
					\item \gls{complan}
					\item \glspl{fscm}
					\item Positions des \gls{farp}
				\end{e5}
			\end{e4}
		\end{e3}
	\end{e2}
\end{e1}
