\section{Introduction}

\begin{e1}
	\item La préparation comprend toutes les activités effectuées par l'unité pour améliorer sa capacité à effectuer des opérations, parmi lesquelles: répétitions, mouvements et observations (\fullref{preparation}).
	\item Une fois le plan établi et approuvé, il doit être répété. Cela inclut les circuits de communications primaires et secondaires, et a méthodologie de contrôle. Les observateurs doivent être identifiés et leurs capacités de communication vérifiées. Les mouvements tactiques sur le champ de bataille doivent être pris en compte. \textbf{Le plan dans son ensemble doit être réalisable, exécutable et tactiquement sain}.
	\item La coordination entre les différents échelons et les préparations juste avant l'exécution sont tout aussi importantes que le développement du plan lui même. Pour le Staff, la préparation inclut le rassemblement et la mise à jour constante des estimations (\gls{jipoe})
	\item Se préparer implique des briefings \gls{coe}, des missions de répétition \gls{coe}, l'\gls{opord}, des brief-back, la vérification de l'équipement et des communications, la révision des \glspl{sop}, la vérification du plan de chargement, les vérifications avant-combat, et le test de l'armement.
\end{e1}

\efig{preparation}{Préparation au CAS}