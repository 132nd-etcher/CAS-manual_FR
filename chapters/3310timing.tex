\section{Timing}

\begin{e1}
	\itemt{Temps disponible}{
		Le temps nécessaire à la planification de la mission doit être estimé de manière réaliste, et ce temps doit être lui-même implémenté dans la planification.}
	
	\itemt{Cycle de planification}{
		Si une opération \gls{cas} prévue n'est pas prête au moment de son exécution prévue, elle sera d'office réassignée comme \gls{req} immédiate.}
	
	\itemt{Synchronisation}{
		Pas d'application dans DCS.
	}
	\note{
		Lors d'opération sous DCS, on distingue trois façon de donner un temps:
		\begin{e2}
			\itemt{Heure réelle (temps \gls{irl})}{
				Il s'agit de l'heure réelle, comme indiquée à nos montres.
				
				Cette heure est ajustée à la saison (heure d'été, heure d'hiver) et dépend du fuseau horaire.
			}
			
			\itemt{Heure réelle GMT (temps ``ZULU'', prononcé ``zoulou'')}{
				L'heure GMT est l'heure réelle, dans le fuseau horaire de Greenwitch.
				
				Il s'agit d'une heure de référence pour toute la planète, qui permet de se synchroniser lorsqu'on travaille avec des gens qui se trouvent dans un fuseau horaire différents.
			}
			
			\itemt{Heure dans le jeu (temps \gls{ig})}{
				L'heure \gls{ig} est l'heure simulée dans DCS, affichée sur l'horloge de bord dans le cockpit, et généralement donnée avant la mission par le créateur de mission.
				
				Cette heure détermine les conditions d'illumination (jour/nuit) de l'environnement simulé.
			}
			
			\item Les fuseaux horaires sont indiqués par une lettre de l'alphabet, prononcée comme prescrit dans l'alphabet phonétique:
			
			Z: heure GMT.
			A: heure GMT+1h (heure d'hiver en France).
			B: heure GMT+2h (heure d'été en France).
			\ldots{}
			D: heure GMT+4h (heure d'hiver en Géorgie).
			\ldots{}
			
			\item Les heures pourront être référencées comme suit:
			
			\begin{e3}[4em]
				\item 20h IG:  8 heure du soir en jeu.
				\item 2000IG:  8 heure du soir en jeu.
				\item 2000IRL: 8 heure du soir dans la réalité.
				\item 2000A:   8 heure du soir dans le fuseau horaire GMT+1.
				\item 2000Z:   8 heure du soir GMT.
			\end{e3}
			
		\end{e2}
	}
\end{e1}