\section{Minimiser le tir fratricide}

\begin{e1}
	\itemt{Général}{Le tir fratricide est parfois le résultat tragique de la guerre.}
	
	\item{Causes}{
		Bien que parfois le résultat d'armement défectueux, le tir fratricide est souvent le résultat de la confusion qui règne sur le champ de bataille.
		
		\note {Kakane, celle-là est pour toi !}
		
		Les causes peuvent être, entre autres:
		\begin{e2}
			\item Mauvaise identification de la cible.
			\item Position ou description de la cible erronée.
			\item Mauvaise transmission de la position ou de la description de la cible.
			\item Perte de \gls{sa} par le pilote ou le \gls{jtac}.
		\end{e2}
		
		Pour réduire le risque de tir fratricide, on peut employer:
		\begin{e2}
			\item Planification de mission détaillée.
			\item Procédures standardisées.
			\item Entraînements et répétitions réalistes.
			\item Utilisation d'appareils de marquages ou de tracking des unités alliées.
			\item Staff efficace.
			\item \gls{faca}.
			\item Autorisations de tir réfléchies.
		\end{e2}
	}
	
	\itemt{Responsabilités}{
		Tout le personnel impliqué dans le \gls{cas} est responsable de la sécurité lors du planning et de l'exécution. Chaque participant doit tout entreprendre pour identifier les unités alliées, les forces ennemies et les civils avant de prendre pour cible, d'engager, ou d'autoriser le tir.
		
		Le \glsfull{cid} est le processus par lequel on obtient une classification des objets détectés dans un environnement de combat suffisamment satisfaisante que pour autoriser un engagement.
		
		En fonction de la situation, cette classification peut être limitée à ami, ennemi, neutre ou non-combattant, ou inclure des informations supplémentaires comme la classe, le type, la nationalité ou la configuration.
		
		La \gls{cid}, en corrélation avec les \glspl{roe}, permettent de décider d'autoriser ou d'empêcher l'utilisation d'armement létal.
	}
	\itemt{Entraînement}{La coalition, ses composantes, et les unités doivent effectuer un entraînement conjoint régulier qui simule des opérations réelles de manière à développer les capacités nécessaires au bon déroulement du \gls{cas}.
	}
	
\end{e1}

