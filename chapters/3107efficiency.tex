\section{Efficacité du CAS}

\begin{e1}
	\item 
	Pour que le \gls{cas} soit efficace, il faut:
	\begin{e2}
		\itemt{Du personnel correctement entraîné}{
			L'entraînement au \gls{cas} doit intégrer toutes les manoeuvres et procédures nécessaires à l'exécution du \gls{cas}. Cet entraînement doit être tenu à jour.
		}
		
		
		\itemt{Un planning et une intégration réfléchie}{
			Un \gls{cas} efficace s'appuie sur une planification en profondeur. La capacité à fournir la puissance de feu nécessaire au bon endroit est possible de par l'intégration avec les forces au sol. De point de vue du planificateur, le \gls{cas} pré-planifié sera toujours préféré.
		}
		
		\itemt{Un \gls{cc} efficace}{
			Le \gls{cas} nécessite une structure \gls{cc} flexible et intégrée au reste du dispositif pour trier les demandes, assigner les priorités et les tâches, repositionner les éléments, fournir les alertes de menaces, et augmenter les capacité de \gls{cid}.
			
			Pour ce faire, le \gls{cc} a besoin d'une communication fiable entre les pilotes, les agences de contrôle, les \glspl{jtac}, les forces au sol, et les \glspl{gc}.
			
			De par l'utilisation mesurée des \glspl{acm} et des \glspl{fscm}, les \glspl{gc} peuvent faciliter l'emploi du feu dans leur zone.
			
			Cfr. \citetitle{jp352} pour plus d'explications concernant les \glspl{acm}.
			
			Cfr. \citetitle{jp309} pour plus d'explications concernant les \glspl{fscm}.
			
			Cfr. \citetitle{jp330} pour plus d'explications concernant \gls{cc} en opération Joint.
		}
		
		\item La supériorité aérienne (tout particulièrement le \gls{sead}).
		
		\itemt{Marquage des cibles}{
			Fournir des marques appropriées et au bon moment permet d'améliorer grandement l'efficacité du \gls{cas}. Le marquage augmente la \gls{sa}, permet d'identifier une cible unique parmi d'autres, diminue le risque de tir fratricide et de dommage collatéral, et facilite le \gls{tac}.
			
			Si un \gls{gc} prévoit tomber à court de marques, il devra le spécifier lors de la phase de planification.
			
			\fullref{chap3} pour plus d'informations à propos de la marque.
		}
		
		\item Une reconnaissance et une connaissance des cibles
		
		\itemt{Des procédures flexibles et éprouvées}{
			Un appui-feu réactif permet au \gls{gc} de répondre aux changements rapides du champ de bataille. Comme le champ de bataille moderne est extrêmement dynamique, les procédures \gls{cas} doivent permettre de changer de cible, d'arme ou de tactique très rapidement. Le demandeur est souvent le plus à même de déterminer les besoins en termes d'appui-feu, et comme tout appui-feu, le \gls{cas} se doit d'être réactif pour être efficace.
			
			Certaines techniques peuvent être utilisées pour améliorer le temps de réponse du \gls{cas}:
		}
		\begin{e3}
			\item Placement des unités de \gls{cas} proche de leur zone d'opération. Placement des points d'attente de d'orbite de manière optimiser le temps de réaction.
			\item Délégation des autorisations de tir aux unités subordonnées.
			\item Ré-assignation des appareils en fonction des situations émergentes.
			\item Révision de l'\gls{ato} en fonction des situations émergents.
			\item Redirection des appareils \gls{cas} en réponse aux menaces.
			\item Délégation des autorité au plus bas niveau tactique possible.
			\item Placement des \gls{jtac} avec les unités au sol de manière à faciliter les communications et la coordination avec les appareils de \gls{cas}.			
		\end{e3}
		
		\itemt{De l'armement approprié}{
			Le \gls{gc} et le \gls{jtac} doivent connaître l'effet de l'armement employé, de manière à pouvoir évaluer les possibilités de dommages collatéraux, et l'impact sur le poursuite de la mission.
			\item Des conditions environnementales.
			
			Des conditions favorables améliorent l'efficacité des unités de \gls{cas} quelles que soient les capacités de l'appareil ou de l'armement. \textbf{Avant d'envisager une mission de \gls{cas}, des conditions météo minimales doivent être définies}. Certaines conditions peuvent n'influencer que certaines plateformes; par exemple, un plafond très bas peut rendre certains \glspl{fw} inutiles, sans impacter les \glspl{rw}. Inversement, les \glspl{fw} pourront opérer lors d'une tempête de sable qui clouerait les \glspl{rw} au sol. Les conditions météo influencent également fortement le marquage de la cible.
		}
		
		\efig{caswwi}{Le CAS pendant la première guerre mondiale.}
	\end{e2}
\end{e1}

