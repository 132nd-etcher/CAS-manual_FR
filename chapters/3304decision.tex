\section{Le CAS dans le processus de prise de décision}

Le processus de prise de décision \gls{cas} est un cycle continu, en trois phases, qui se concentre sur les aspects appui-feu et \gls{cas} de l'opération.

Ce processus aide les commandants et leur staff à développer la portion \gls{cas} du plan d'appui-feu.

Ce chapitre se concentre sur la phase de planification, décrite dans la \fullref{planningphase}.

La phase d'exécution est décrite dans le \smallref{chap5}.

\remark{
	Dans le cadre de ce document les ``Planificateurs'' sont:
	
	\begin{itemize}
		\item Officiers staff pour l'appui-feu.
		\item Officier S3 (opérations) du Bn, de la brigade ou de la division.
		\item Les \glspl{alo}.
	\end{itemize}
}

\efig{planningprocess}{Processus de planification extrait du \citetitle{jp50}.}

\efig{integrationmodel}{Modèle d'intégration du CAS dans le processus de prise de décisions.}

\efig{planningphase}{Phase de planification.}

La phase de planification commence lors de la réception des ordres, et se termine lors de la diffusion des ordres de mission \gls{cas} aux unités subordonnées.

Elle peut être décomposée en 5 étapes, décrites dans les sous-sections suivantes.

%\begin{e1}
%	\item Le \gls{req} est effectué par l'unité qui demande un soutien aérien. On distingue deux principaux types de demandes:
%	\begin{e2}
%		\item Le \gls{cas} planifié à l'avance (``pre-planned \gls{cas}'', ou simplement ``\gls{cas}'')
%		\item Le \gls{cas} ``à la demande'' (``on-call cas''), qui regroupe deux possiblités:
%		\begin{e3}
%			\item Le \gls{xcas}, où l'appareil en attente de tasking se trouve déjà dans les airs
%			\item Le \gls{gcas}, où l'appareil en attente de tasking se trouve encore au sol
%		\end{e3}
%	\end{e2}
%	\item Un \gls{req} réel se compose d'un grand nombre d'informations, utiles à la planification de la mission et à la répartition des effectifs disponibles.
%	\item
%	Souvent, pour DCS, l'équivalent du \gls{req} sera contenu dans le briefing de la mission (zone d'opération, menace, composition et force de unités ennemies, timing, emport, unités amies, \gls{jtac}, plan de fréquences, etc…).
%	\item Toute référence ultérieure au \gls{req} fera donc implicitement référence au briefing de mission.
%	\item Les considération à prendre en compte lors de l'établissement d'une \gls{req} sont propres au rôle de \gls{gc}/\gls{jfc} et ne seront pas traitées dans ce document.
%	\item Le \gls{cas} planifié est un \gls{cas} effectué après une \gls{req} spécifique, concernant une zone connue, pour atteindre un objectif déterminé.
%	\item Dans le cas d'un \gls{cas} planifié, un grand nombre de paramètres sont connus avant le décollage, ce qui permet une meilleure préparation.
%\end{e1}

\subsection{Étape 1: réception de l'ordre de mission}

\begin{e1}
	\item
	En tant que participants au processus de planification, les officiers en charge (chef de patrouille, commandant d'escadron, \ldots) doivent être à même de comprendre et analyser l'information à partir de ces différentes sources:
	\begin{e2}
		\item \gls{aob}
		\item \gls{ato}
		\item \gls{aco}
		\item \gls{spins}
		\item \gls{opord}
		\item \gls{sop}
	\end{e2}
\end{e1}
\note{
	Les \glspl{sop} sont propres à l'escadron, et sont normalement connues par les tous les officiers supérieurs.\par
	
	Exceptées les \glspl{sop}, tous les éléments sont normalement fournis sous une forme ou l'autre dans le briefing de mission.\par
	
	S'il devait manquer un élément, il appartient à l'officier en charge de la patrouille d'évaluer son importance et de poser les questions nécessaires à l'organisateur de la mission.\par
}

\subsection{Étape 2: analyse de la mission} 

\begin{e1}
	\item Avant de pouvoir préparer la mission, les officiers en charges devront:
	\begin{e2}
		\item Mettre à jour les différentes sources d'informations (\gls{ato}, \gls{spins}, \gls{aco}, etc.).
		\item Estimer les capacités des forces à leur disposition (équipement, personnel, restrictions, etc.).
		\item Déterminer les capacités et limitation du personnel et de l'équipement à leur disposition.
		\item Fournir leurs inputs au \gls{gc}.
		\item Déterminer les tâches essentielles, spécifiques et implicites de la mission.
		\item Évaluer les conditions relatives à:
		\begin{e3}
			\item L'ennemi.
			\item La météo.
			\item Le terrain.
			\item Unités et support disponibles.
			\item Temps disponible.
		\end{e3}
		\item Avertir les unités (personnes) subordonnées.
		\item Anticiper la quantité de support aérien nécessaire en se basant sur:
		\begin{e3}
			\item Les priorités du \gls{hhq}.
			\item Les faits et les estimations.			
		\end{e3}
		\item Diffuser les documents/informations suivants:
		\begin{e3}
			\item Estimations de l'\gls{alo}.
			\item Unités \gls{cas} disponibles.
			\item Restrictions et contrainte relatives au \gls{cas} (temps de réponse, limitations dues à la météo, directives tactiques, \glspl{roe}, etc.).
			\item Avertissements aux unités subordonnées.
			\item Vérification de la capacité des \glspl{tacp} à fournir un support à la mission.
		\end{e3}
	\end{e2}
	
	\itemt{Points clef:}{
		Lors de la phase de planification, les planificateurs devront être familiers avec les éléments suivants:
	}
	
	\begin{e2}
		\item \gls{conops}: Quelles sont les intentions du \gls{jfc}? Attaque ou de défense? Surprise ou délibéré? \glspl{roe}?
		\item Comment s'intègre le \gls{cas} au reste du dispositif?
		\item Quelles pourraient être les intentions de l'ennemi? Comment ces intentions sont-elles affectées par le terrain, la météo, l'heure ?
		\item Quels sont les moyens de surveillance ou de reconnaissance à notre disposition?
		\item \gls{complan} Comment vont s'organiser les communications? Est-ce que toutes les unités participantes au \gls{cas} sont intégrées au \gls{complan}  de manière fiable et redondante?
	\end{e2}
	
	\itemt{Demande de \gls{cas} planifiée}{
		Une fois que les planificateurs ont analysé la mission et sont familiers avec ses implications, la \gls{req} initiale devrait être rédigée (Cfr. \cruderef{datacard} pour le format de la \gls{req}).
		
		Dans le cas d'une demande de \gls{cas} pré-planifié, un grand nombre d'informations supplémentaires pourront être obtenues via la demande elle-même. Par exemple:
	}
	
	\begin{e2}
		\item Zone de \gls{cas}
		\item Menaces
		\item Type de cibles
		\item Localisation de la cible
		\item Localisation des forces amies
		\item Type de terrain
		\item Restrictions horaires
		\item \gls{jtac}
	\end{e2}
\end{e1}

\subsection{Étape 3: développement de la ligne de conduite}

Après voir reçu les directives initiales, le staff prépare la \gls{coa} qui fournira une solution/méthode potentielle pour accomplir la mission.

Le staff crée plusieurs \glspl{coa}, de façon à donner offrir différents choix au \gls{gc}.

Durant cette étape, les planificateurs:

\begin{e1}
	\item Mettent à jour les différentes publications (\gls{ato}, \gls{aco}, \gls{spins}, \ldots{}).
	\item Analysent le rapport de force entre les appareils \gls{cas} disponibles et les forces au sol ennemies, tout particulièrement la menace sol-air.
	\item Produisent des options à inscrire dans la \gls{coa}. Les options sont des activités qui peuvent être effectuées pour atteindre l'objectif. Les options, et les groupes d'options, ainsi que leur branchement, permettent au \gls{gc} de réagir rapidement au changement en cours d'opération ou de campagne.
	\item Déterminent les pré-requis pour le \gls{cas}.
	\item Développent un plan d'appui-feu et un plan d'occupation de l'espace aérien.
	\item Coordonnent l'activation des \glspl{fscm}/\glspl{acm}.
	\item Développent le plan d'intégration du \gls{cas} en fonction du placement des \glspl{tacp}.
	\item L'\gls{alo} assistent au développement des:
	\remark{
		\begin{e2}
			\item Zones d'engagement.
			\item \glspl{tai}.
			\item Déclencheurs.
			\item Zones cibles.
			\item Plan de mouvement.
		\end{e2}
	}
	\item Préparent la diffusion de la \gls{coa} et graphiques associés. Cette étape implique une réflexion quant à la façon d'amasser la plus grande puissance de feu possible contre l'ennemi (\gls{cas}, \gls{ewr}, \gls{isr}, et \gls{idf}).
	
	\itemt{Points clef:}{
		Pour chaque \gls{coa}, les planificateurs prennent en compte:
	}
	\begin{e2}
		\item Intentions du \gls{gc}: comment le \gls{gc} souhaite-t-il utiliser le \gls{cas}? Quels sont les objectifs? Est-ce que le \gls{cas} aide le \gls{gc} à  accomplir sa mission?
		\item \glspl{ccir}: quelles \glspl{ccir} peuvent fournir les appareils de \gls{cas}? Est-ce que les \glspl{tacp}, \glspl{jfo} et/ou \glspl{faca} peuvent fournir des informations importantes? Comment ces informations seront-elles relayées à l'unité qui manoeuvre?
		\item Situation de l'ennemi: où se trouvent les ennemis et comment se battent-ils? Où vont-ils? Où puis-je les engager ou les empêcher d'agir? Quand seront-ils à cet endroit? Que peuvent-ils faire pour me détruire ou déranger mes plans? Comment vais-je les détruire ou les influencer?
		\item Communiqué et sketches: une fois les \glspl{coa} déterminées, un sketch devrait être établir pour chacune d'entre elles, avec des annotations. Ces sketch décriront:
		\remark{
			\begin{e3}
				\item Points de \gls{stack}.
				\item \glspl{ip}.
				\item Points d'entrée/sortie de la zone opérationnelle.
				\item Positions et \glspl{aof} de l'artillerie.
				\item Est-ce que le plan permet l'engagement simultané par le \gls{cas} et les unités au sol?
				\item Est-ce que le plan a été diffusé aux autres unités?
				\item Où se trouveront les \glspl{jtac}/\glspl{jfo}?
				\item De quels \glspl{fscm}/\glspl{acm} le plan a-t'il besoin?
			\end{e3}
			}
	\end{e2}
	\itemt{\gls{tacp}}{
		Le \gls{tacp} fournit les informations suivantes lors de l'établissement des \glspl{coa}:
	}
	\begin{e2}
		\item Portions spécifique au \gls{tacp}:
		\begin{e3}
			\item Plan d'observation (zone cible, appareils, senseurs, et \gls{bda}).
			\item Emploi (par ex., \glspl{aca}).
			\item \gls{complan}.
		\end{e3}
		\item Évaluation des capacités/limitations du \gls{tacp}:
		\begin{e3}
			\item Personnel.
			\item Équipement.
		\end{e3}
		\item Procédures préconisées.
		\item Dernière mise à jour des informations du renseignement.
	\end{e2}
\end{e1}

\subsection{Étape 4: analyse COA/War Game}

Sous-section intentionnellement vide.

\subsection{Étape 5: productions des ordres de mission}

Le staff prépare les ordres de mission contenant la \gls{coa} retenue.

Ces ordres contiennent toutes informations dont ont besoin les unités subordonnées pour remplis leur mission.

Les \glspl{tacp} produisent les portions spécifiques au \gls{cas}:

\begin{e1}
	\itemt{But du \gls{cas}}{
		Ce qui doit être accompli par le \gls{cas}. Fournit des directives pour l'engagement et l'attaque, et la façon dont le \gls{cas} sera synchronisé avec le \gls{cc}, le renseignement, le mouvement, la protection et le maintien des troupes alliées.
	}
	
	\itemt{Priorité}{
		Établit les priorités pour le \gls{cas}, et la façon dont ces priorités évoluent lors des différentes phases de l'opération.
	}
	
	\itemt{Allocation}{
		Détermine l'allocation des unités d'appui-feu: cible allouées, sorties \gls{cas}, \gls{idf} disponible pour la marque fumigène, cibles prioritaires, cibles nécessitant des munitions spéciales, et équipes d'observation équipées laser/\gls{ir}.
	}
	\itemt{Restrictions}{
		Traite des \glspl{fscm} et de l'usage de certaines munitions (illumination, fumigène, mines, fragmentation, \ldots{}).
	}
\end{e1}

Les ordres de mission traitent également les différentes \gls{acm} en vigueur sur le théâtre d'opération.

\begin{e1}
	\item Le ``Mission planning'', ou ``préparation de la mission'', est une étape cruciale au bon déroulement de cette dernière.
	\item Cette préparation sera effectuée par le chef de patrouille, qui sera chargé de rassembler et analyser toutes les informations à sa disposition pour préparer au mieux la mission de la patrouille dont il a la charge.
	\item Lorsque c'est possible, la préparation de la mission se fait en collaboration avec les différents éléments impliqués, certainement avec le \gls{jtac} en charge de la zone attribuée à la patrouille, le \gls{tacon}, et les autres chefs de patrouilles opérant dans le même théâtre opérationnel.
	\item Le chef de patrouille veillera tout particulièrement à toujours utiliser les informations les plus récentes, et fera attention à se tenir au courant de l'évolution de la situation.
	\item Il devra également s'assurer que tous les éléments sous son contrôle ont reçu et compris ses intentions pour la mission.
	\item Checklist de préparation de mission. \fullref{ann1}
\end{e1}



