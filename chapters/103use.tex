\section{Usage du CAS}

Le \glsfull{cas} est utilisé pour attaquer l'ennemi dans un dans le cadre du \gls{conops}, dans un grand nombre d'environnements, de jour comme de nuit, et comme appui-feu. La vitesse, la portée et la manoeuvrabilité des unités aériennes leur permettent d'attaquer des cibles hors de portée des unités au sol ou navales conventionnelles.

Les \glspl{gc} sont l'autorité ultime pour l'usage de la force dans leur zone de contrôle, et décident des cibles prioritaires, des effets qu'ils souhaitent, et du timing du \gls{cas}.

\begin{e1}
	
	\itemt{Utilité sur le champ de bataille}{
		Le \gls{cas} donne aux \glspl{gc} un appui-feu flexible et réactif.
		
		En utilisant le \gls{cas}, les \glspl{gc} peuvent mieux tirer parti des opportunités qui se présentent sur le champ de bataille, en concentrant la puissance de feu pour pousser un avantage tactique, ou pour réduire les risques tactiques et opérationnels. Le mobilité de la vitesse des forces aériennes donnent aux \glspl{gc} un moyen de frapper l'ennemi rapidement et de manière inattendue.
	}
	
	\itemt{Critères pour l'usage du \gls{cas}:}{}
	
	\begin{e2}
		\item Mission et \gls{conops}.
		\item Disposition, force et composition de l'ennemi.
		\item Capacités et limitations des appareils engagés dans le \gls{cas}.
		\item Emplacement et équipement des \gls{jtac}.
		\item \glspl{roe}.
		\item \gls{spins}.
		\item Défenses anti-aériennes ennemies et capacités alliées à les contrer.
		\item Disposition des forces amies
		\item Allocation des sorties \gls{cas}.
		\item Emplacement des civils et estimation des dommages collatéraux potentiels.	
	\end{e2}
	
	\item Les cibles du \gls{cas} sont sélectionnées par le \gls{jtac}, en commençant analyser les intentions du \gls{gc}, puis en fonction du terrain, de la météo, de la mission, des défenses ennemies, de l'armement disponible, du temps de réponse, des capacités ou limitations propres aux appareils de \gls{cas} présents, etc. D'autres considérations sont également à prendre en compte, comme la géométrie d'attaque, la proximité des forces amies, les dommages collatéraux potentiels, la capacité des senseurs, et l'appui-feu supplémentaire disponible.
	
	Les pilotes gardent la responsabilité d'émettre des recommandations quant à l'usage optimal de l'armement et l'emploi de tactiques spécifiques, pendant que le \ja{} se concentrent sur l'obtention de l'effet désiré (\fullref{chap3} pour les tactiques d'emploi des appareils).
	
	Les \glspl{gc}, \glspl{jtac}, \glspl{faca} et les pilotes peuvent utiliser les \glspl{bda} pour déterminer les objectifs ont été remplis, ou si une nouvelle attaque est nécessaire.
\end{e1}
	
