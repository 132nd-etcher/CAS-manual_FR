%\chapter{Carte de mission}\label[annex]{ann2}
\invisiblechapter{Authentification DRYAD}\label[annex]{dryad}

Les informations officielles à propos de l'authentification DRYAD sont classifiées.

Cfr. \citetitle{wikipediadryad}.

Une table d'authentification DRYAD se trouve en \smallref{dryadauth}.

L'authentification DRYAD fonctionne de la manière suivante:

\begin{lstlisting}[caption=Authentification DRYAD., label=exdryad]
	MAGIC, ici REDWOLF
	REDWOLF, MAGIC, authentifiez-vous Charlie, Mike, Echo.
	MAGIC, ici REDWOLF, je m'authentifie Echo, authentifiez-vous Juliet, Papa, Delta.
	REDWOLF, MAGIC, je m'authentifie Papa.
\end{lstlisting}

Fonctionnement par l'exemple:
\begin{e1}
	
	\item \textbf{Remarque: les lignes avec l'alphabet dans l'ordre et les chiffres en dessous sont des en-têtes et doivent être ignorées}
	
	\tfig{90063dryad}{Lignes à ignorer}{dryadignore}

	\item 
	
	MAGIC demande à REDWOLF de s'authentifier Charlie, Mike, Echo (CME).
	
	\begin{e2}
		\itemt{Notions importantes:}{
			
			Pour cette explication, on définit les termes suivants:}
			\begin{e3}
				\item Première lettre: Charlie.
				\item Seconde lettre: Mike.
				\item Troisième lettre: Echo.
				\item Première ligne: ligne correspondant à la première lettre.
				\item Seconde ligne: ligne que l'on aura trouvée après avoir effectué le première correspondance (cfr. ci-dessous).
			\end{e3}
	\end{e2}
	
	REDWOLF consulte sa table DRYAD et chercher la première lettre, Charlie, dans la colonne tout à gauche.
	
	Cette lettre correspond à une \textbf{ligne}, que voici:
	
	\tfig{90062dryad}{Première ligne}{dryadfirstline}
	
	\item Une fois la première ligne trouvée, on y cherche la seconde lettre, Mike:
	
	\tfig{90064dryad}{Seconde lettre trouvée dans la première ligne}{dryadsecondletter}
	
	\item Maintenant on regarde la lettre qui se \textbf{en dessous} de Mike:
	
	\tfig{90065dryad}{Lettre en dessous de la seconde lettre}{dryadletterbelowsecond}
	
	Dans cet exemple, la lettre en dessous est Yankee.
	
	\item Maintenant, on cherche la ligne correspondant à Yankee:
	
	\tfig{90066dryad}{Seconde ligne}{dryadsecondline}
	
	\item Une fois la seconde ligne trouvée, on y cherche la troisième lettre, Echo:
	
	\tfig{90067dryad}{Troisième lettre}{dryadthirdletter}
	
	\item Enfin, dernière étape, on cherche la lettre qui se trouve en dessous, et ce sera la réponse à la demande d'authentification:
	
	\tfig{90068dryad}{Réponse}{dryadanswer}
	
	\item Résumé de toutes les étapes ci-dessus:
	
	\tfig{90069dryad}{Résumé authentification DRYAD}{dryadsummary}
	
\end{e1}

\newpage

%\thispagestyle{logoonly}
%\fancyhead[L]{\minipage[b]{.9\linewidth}\large Table d'authentification DRYAD \endminipage}

%\begin{adjustbox}{width=\textwidth, max height=\textheight}
%\makebox[\linewidth][c]{\fontsize{20}{20}\selectfont \textsc{Table d'authentification DRYAD}}
\tfig{90061dryad}{Authentification DRYAD}{dryadauth}
\important{Cette table DRYAD est fictive et peut être utilisée à des fins de simulation.}
%\end{adjustbox}