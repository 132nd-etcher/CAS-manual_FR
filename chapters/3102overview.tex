\section{Aperçu}

\begin{e1}
	\itemt{Le \acrfull{cas} est une action aérienne effectuée par les \glsfullpl{fw} et des \glsfullpl{rw} contre des cibles hostiles qui se trouvent à proximité de forces amies et qui nécessite une intégration minutieuse des missions aériennes avec le feu et le mouvement des forces amies.}{
		Le \gls{cas} demande une planification en détail, de la coordination et de l'entraînement.
		
		Le \gls{gc} soutenu par le \gls{cas} établit les priorités, les effets et le timing du \gls{cas} dans les limites de sa zone de contrôle. Le \gls{cas} donne la capacité au \gls{gc} d'employer la force aérienne pour détruire, empêcher ou neutraliser les forces ennemies et permet en conséquence le mouvement, la manoeuvre et le contrôle de territoire, de population et de zones maritimes.
		
		Le \gls{cas} est planifié et exécuté de manière à soutenir les unités au sol tactiques. Il est étroitement intégré au niveau tactique. L'allocation des appareils qui effectuent le \gls{cas} se fait au niveau opérationnel (stratégique). La planification du \gls{cas} vie à fournir un appui-feu précis et rapide aux forces amies au contact avec l'ennemi.
		
		Chaque service s'organise, s'entraîne et s'équipe pour employer le \gls{cas} selon son rôle dans la coalition. Un grand nombre d'appareils différents sont capables d'effecteur le \gls{cas}. Le \gls{jfc} et son staff ont la responsabilité d'intégrer ces capacités \gls{cas} dans leur \gls{conops}.}
	
	\itemt{Le \gls{cas} est planifié et exécuté pour appuyer les unités amies au sol.}{
		Le \gls{cas} est étroitement intégré au niveau du contrôle tactique des forces amies supportées. Alors que l'affectation des ressources aériennes disponibles se fait au niveau stratégique opérationnel, le processus de planification du \gls{cas} se fait au niveau stratégique local, pour fournir aux unités amies au contact de l'ennemi un appui-feu précis et rapide.
	}
	
	\itemt{Le \gls{cas} peut être effectué partout ou les forces amies se trouvent au contact de l'ennemi.}{
		Le mot ``rapproché'' (``Close'') n'implique pas une distance spécifique, mais plutôt un contexte. \emph{Le facteur déterminant} est la nécessité d'intégrer en détail le \gls{cas} du fait de la proximité des troupes amies. Parfois, le \gls{cas} peut être le meilleur moyen d'exploiter une opportunité tactique en défense ou en attaque. Le \gls{cas} fournit un appui-feu capable de détruire, perturber, interdire, empêcher, harceler, neutraliser ou retarder l'ennemi.
	}
	\itemt{Chaque organisation s'entraîne et emploie le \gls{cas} en tant que partie d'une coalition.}{
		Le \gls{jfc} est responsable d'intégrer ces organisations au \gls{conops} en fonctions de leurs capacités.
	}
	\itemt{Le \gls{tac} est l'autorité responsable de manœuvrer les appareils de soutien aérien et d'autoriser l'attaque finale.}{
		Un \gls{jtac} ou un \gls{faca} certifié sera reconnu comme capable et autorisé à effectuer le \gls{tac}.
		
		\fullref{controltypessection} pour un description plus détaillée des types de contrôle.
		
		Il y a trois types de contrôle (type 1, 2 et 3):
	}	
	\begin{e2}
		\itemt{\glsfull{typeone}}{
			Le \gls{typeone} est utilisé lorsque le \gls{jtac}/\gls{faca} a besoin de contrôler de chaque attaque individuellement ainsi que d'avoir en permanence en visuel l'appareil qui attaque ainsi que sa cible.
		}
		
		\itemt{\glsfull{typetwo}}{
			Le \gls{typetwo} est utilisé lorsque le \gls{jtac}/\gls{faca} a besoin de contrôler chaque attaque mais qu'il ne peut pas voir l'appareil au moment du largage ou qu'il ne peut pas voir la cible.
		}
		
		\itemt{\glsfull{typethree}}{
			Le \gls{typethree} est utilisé lorsque le \gls{jtac}/\gls{faca} a besoin de contrôler chaque engagement, avec ses restrictions, mais qu'on engagement peut être composé de plusieurs attaques.
		}
		
	\end{e2}
	
	\itemt{\glsfull{tac}}{Le \glsfull{tac} est l'autorité qui contrôle la manoeuvre des appareils de \gls{cas} et qui autorise le tir de munitions.}
	
	\itemt{\glsfull{tgo}}{
		La guidance terminale diffère \gls{tac}. Les \glspl{tgo} sont les actions qui fournissent des informations additionnelles quant à la position de la cible, par des moyens électroniques, mécaniques, visuels ou vocaux.
		
		A moins qu'il aie la qualification de \glsfull{jtac}, le personnel qui effectue la \gls{tgo} n'a pas l'autorité pour contrôler la manoeuvre des appareils de \gls{cas}, ou pour autoriser le tir de munitions.
	}
		
\end{e1}
	
