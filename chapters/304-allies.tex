\newpage
\section{Troupes au sol alliées}

Le planificateur doit prendre en compte les unités \gls{cc}, \gls{isr}, \gls{ew} et \gls{cas}.

\e
	\itemt{Unités \gls{cc}}{
	Un plan \gls{cc} détaillé, flexible et redondant est essentiel. Des unités \gls{cc} aéroportées peuvent faciliter le travail du \gls{cc}, mais chaque plateforme a ses capacités et limitations qui doivent être prises en compte lors du planning. Au minimum, les points suivants doivent être évalués:}
	\ee
		\itemt{Unités \gls{cc} aéroportées}{
		Est-ce que ces unités ont une importance critique pour la mission ou seulement pour une partie de celle-ci? Quel sont le rôle et la fonction de chacune? Est-ce que le nécessaire a été fait pour assurer la communication entre le \gls{cc} et les \gls{rw} volant bas?}
		\itemt{Unités \gls{cc} au sol}{
		Intégrer les unités \gls{cc} au sol revêt une importance capitale. Quel sont le rôle et la fonction de chacune? Est-ce que le nécessaire a été fait pour assurer les communications avec le \gls{cc}?.}
	\ed
	\itemt{Unités \gls{isr}}{
	Toutes les sources d'\gls{isr} sont d'excellentes (drone, \gls{elint}, scouts, \gls{faca}, rapports des appareils d'attaque, etc.) sources d'informations.}
	\itemt{Capacités et armement des appareils \gls{cas}}{
	Le planificateur choisit les combinaisons d'appareils et d'armement qui offrent la précision, la puissance et la flexibilité nécessaires pour obtenir l'effet désiré. Le \gls{gc} doit fournir suffisamment d'informations à propos de cet effet désiré, ainsi que toute restriction tactique ou limitation. Les \gls{gc} doivent savoir qu'une \gls{req} immédiate risque d'être assurée par un appareil ne disposant pas de l'armement optimal pour la mission.}
\ed