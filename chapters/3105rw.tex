\section{Emploi des voilures fixes et des voilures tournantes en opérations de CAS}

La structure organisationnelle, la mission principale et les capacités des appareils capables d'effectuer le \gls{cas} déterminent leur emploi. Bien que les \glspl{fw} et les \glspl{rw} puissent tous deux effectuer le \gls{cas}, leur emploi diffère.

\begin{e1}
	
	\itemt{Considérations d'ordre général}{
		L'emploi des appareils se fait en \emph{sorties}. Une sortie équivaut à \emph{un vol} pour \emph{un appareil}. Généralement, les appareils \gls{cas} volent par groupe de deux ou quatre, c-à-d deux à quatre sorties.
	}
	
	\itemt{Considérations pour les \glspl{fw}}{
		De par leurs vitesse et portée, les \glspl{fw} donnent au \gls{jfc} la possibilité de délivrer une importante puissance de feu à l'endroit et au moment souhaité. De plus, les \glspl{fw} peuvent embarquer un arsenal varié d'armement et de senseurs, leur permettant d'accomplir leur mission par tous les temps et dans des conditions météo difficiles..
	}
	
	\itemt{Considérations pour les \glspl{rw}}{
		Les \glspl{rw} permettent de manoeuvrer et de repositionner rapidement la puissance de feu en fonction de l'évolution de la situation. Ils ont un excellent temps de réponse et peuvent rester longtemps sur zone, évoluer à très faible altitude, et effectuer du \gls{cas} sur tous les types de terrain. Ils offrent également des capacités d'infiltration et d'extraction du personnel.
	}

\end{e1}

