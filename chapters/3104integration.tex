\section{Intégration du CAS}

Lors des opération conjointes, l'intégration du \gls{cas} commence au niveau opérationnel. S'il est établi, le \gls{jfacc} fournit ses recommandations au \gls{jfc}. Chaque composante informe le \gls{jfc} de ses besoins et limitations. Le \gls{jfc} implémente le cadre dans lequel les opérations d'interdiction (\gls{cas}, \gls{ai}, etc.) s'intégreront dans l'\gls{opord}, l'\gls{aod}, l'\gls{ato}, l'\gls{aco} et les \gls{spins}.

Le \gls{jfacc} donne ses recommandation quant à l'allocation des appareils au \gls{jfc} après avoir consulté les composantes subordonnées.

Une fois que le \gls{jfc} a décidé de l'allocation des appareils aux différentes missions (\gls{cas}, \gls{das}, \ldots{}), le \gls{jfacc} taske les appareils de manière à remplir les différentes \glspl{req} reçues.

Le \gls{conops} du \gls{jfc} fournit le cadre pour intégrer les opérations Joint (\gls{cas}, \gls{das}, \ldots{}) dans l'\gls{opord}, les \glspl{aod}, l'\gls{ato}, l'\gls{aco} et les \gls{spins}.

\begin{minipage}{\linewidth}
	\subsection{Battle Rythm}
	
	Le ``Battle Rythm'', ou rythme de la bataille, est un cycle quotidien et continu.
	
	Voici un exemple de ``Battle Rythm'' pour une partie nous intéresse particulièrement, l'\gls{ato}:
	
	\efig{atobattlerythm}{Battle Rythm de l'ATO extrait du \citetitle{jp330}.}
\end{minipage}

\subsection{Création de l'OPORD}

Dans une opération de grande envergure, l'\gls{opord} est un document de référence très important.

Toutes les parties pourront y trouver, tout au long de l'opération, les informations nécessaires à la préparation et à l'exécution de leur mission.

Cfr. \citetitle{attp501}, Chapitre 12, pour un guide à la création et à la lecture d'un \gls{opord}.


