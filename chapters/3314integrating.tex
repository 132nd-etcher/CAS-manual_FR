\section{Intégration du CAS au reste du dispositif}

L'emploi d'appareils de \gls{cas} conjointement avec le tir de surface implique une planification minutieuse combinée avec une capacité de s'adapter rapidement une situation qui évolue.\par

Des techniques de séparation sont utilisées à cette fin.

\subsection{Fire Support Coordination Measures (FSCMs)}
	Les \glspl{gc} implémentent dans leur zone des \glspl{fscm} permissives et restrictives pour accélérer le processus d'engagement des cibles; protéger les unités alliées, les populations, l'infrastructure importante, les sites religieux ou culturels; autoriser le tir Joint; établir les conditions pour préparer une opération future.
	
\begin{e1}
	
	\item La vocation principale des mesures permissives est d'accélérer l'engagement des cibles.
	
	\item La vocation principale des mesures restrictives est de protéger les forces alliées.
	
	\item Pour plus de détails concernant les \glspl{fscm}, cfr. \citetitle{jp309}.
	
	\itemt{Mesures permissives:}{}
	
	\begin{e2}
		
		\itemt{\glsfull{cfl}:} {\glsd{cfl}}
		
		\itemt{\glsfull{fscl}:} {\glsd{fscl}}
		
		\efig{fscl}{Positionnement d'une FSCL (\citetitle{jp303})}
		
		La création d'une \gls{fscl} n'implique pas une \gls{ffa} derrière la \gls{fscl}.
		
		\item BCL - spécifique à l'USMC
		
		\itemt{\glsfull{ffa}:} {\glsd{ffa}}
		
		\itemt{\glsfull{kb}:} {\glsd{kb}}
		
		\efig[\linewidth][0.4\textheight]{bluekb}{Blue Killbox, extrait du \citetitle{jp309}.}
		
		\efig[\linewidth][0.4\textheight]{purplekb}{Purple Killbox, extrait du \citetitle{jp309}.}
		
		\itemt{Statut d'un Killbox} {Une Killbox peut être ``chaude'' (HOT) ou ``froide'' (COLD):}
		
		\begin{e3}
			
			\itemt{HOT:}{L'emploi d'armement est autorisé dans la Killbox sans autre forme de coordination ou de déconfliction.}
			
			\itemt{COLD:}{L'emploi d'armement dans la Killbox n'est \emph{pas} autorisé sans coordination préalable.}
			
		\end{e3}
		
		Cfr. \citetitle{jp309}, Annexe A, pour plus d'informations à propos de la Killbox.
		
	\end{e2}
	
	\itemt{Mesures restrictives:}{}
	
	\begin{e2}
		
		\itemt{\glsfull{nfa}:} {\glsd{nfa}}
		
		\itemt{\glsfull{rfa}:} {\glsd{rfa}}
		
		\itemt{\glsfull{rfl}:} {\glsd{rfl}}
		
		\itemt{\glsfull{aca}:} {\glsd{aca}}
		
		\begin{e3}
			
			\itemt{\gls{aca} formelle}{
				L'autorité de contrôle de l'espace aérien établit l'\gls{aca} à la demande du \gls{gc}.
				
				Les \glspl{aca} formelles demandent une planification détaillée et sont intégrées dans l'\gls{aco}, l'\gls{ato} ou les \gls{spins}.
			}
			
			\efig{formalaca}{ACA formelle.}
			
			\itemt{\gls{aca} informelle}{
				Une \gls{aca} informelle est établie par le \gls{gc} au moyen d'une plan de séparation.
				
				L'\gls{aca} informelle permet d'établir une déconfliction immédiate et temporaire, et, de ce fait, ne durent généralement pas très longtemps.
				
				De par leur nature immédiate, il peut être difficile pour les autorités de contrôle de s'assurer de leur diffusion correcte.
			}
			
		\end{e3}
		
		\itemt{Séparation}{}
		
		\begin{e3}
			
			\begin{minipage}{\linewidth}
				
			\itemt{Séparation latérale}{
				La séparation latérale consiste à délimiter la zone d'engagement en deux (ou plus) parties au moyen d'une ligne.
				
				Cette ligne peut être une coordonnée MGRS, une latitude/longitude, ou un point particulier du relief.
			}
			
			\begin{lstlisting}[caption=ACA: Séparation latérale, label=acalateralsep]
				REDWOLF ici PIRATE, rester "à" l'ouest de la rivi"è"re.
			\end{lstlisting}
			
			\efig{lateralsep}{ACA: Séparation latérale.}
			\end{minipage}
			
			\begin{minipage}{\linewidth}
			\itemt{Séparation en altitude:}{
				La séparation en altitude peut être utilisée lorsqu'un appareil doit croiser la ligne de tir d'une unité alliée.
				
				La séparation en altitude en appelant ``Restez au dessus de \ldots{}'' ou ``Restez en dessous de \ldots{}''.
				
				L'altitude est données en \gls{ft} \gls{msl}, sauf si indiqué autrement.
			}
			
			\efig{altsep}{ACA: Séparation en altitude.}
			\end{minipage}
			
			\begin{minipage}{\linewidth}
			\itemt{Combinaison des séparations latérales et verticales:}{}
			
			\efig{lateralandverticalsep}{ACA: Séparation latérale en altitude.}
			\end{minipage}
			
			\itemt{Séparation horaire:}{
				La séparation horaire est celle qui demande le plus de préparation.
				
				Elle est appropriée lorsque de multiples unités alliées doivent engager la même cible.
				
				Tous les timings se basent sur les \glspl{tot}/\glspl{ttt} des appareils de \gls{cas}.
			}
			\begin{e4}
				\begin{minipage}{\linewidth}
				\itemt{\glsfull{tot}}{
					Le \gls{tot} est l'heure à laquelle les munitions tirées par l'appareil de \gls{cas} sont censées impacter la cible.
					
					C'est la méthode la plus facile à utiliser.
					
					Si le pilote ne peut pas respecter le timing, il doit en informer le \gls{tac} au plus vite.
				}
				
				\efig{tot}{ACA: Séparation horaire avec un TOT.}
				\end{minipage}
				
				\begin{minipage}{\linewidth}
				\itemt{\glsfull{ttt}}{
					Le \gls{ttt} donne un temps en minutes et en secondes entre un moment T (appelé ``HACK'') et le moment où les munitions impactent la cible.
					
					C'est une méthode précise, mais rarement utilisée.
					
					C'est le \ja{} qui donne le HACK, et les pilotes doivent en accuser réception.
				}
				
				\begin{lstlisting}[caption=Synchronisation au moyen d'un HACK, label=hacktime]
					REDWOLF ici PIRATE, Time to target 5+00, pr"ê"t... pr"ê"t... HACK.
					PIRATE ici REDWOLF, roger.
				\end{lstlisting}
				\end{minipage}
				
			\end{e4}
			
		\end{e3}
		
		\efig{fscm}{Illustration des différentes FSCMs.}
		
	\end{e2}
	\begin{minipage}{\linewidth}
	\itemt{\glsfull{acm}:} {\glsd{acm}}
	
	\begin{e3}
		\itemt{\glsfull{ca}:} {\glsd{ca}}
		\itemt{\glsfull{hidacz}:} {\glsd{hidacz}}
		\itemt{\glsfull{roz}:} {\glsd{roz}}
		\itemt{\glsfull{roa}:} {\glsd{roa}}
		\itemt{\glsfull{mrr}:} {\glsd{mrr}}
		\itemt{\glsfull{saafr}:} {\glsd{saafr}}
	\end{e3}
	\end{minipage}
	
	\efig{acm}{ACM Joint.}
	
	Cfr. \citetitle{jp352} pour plus d'informations à propos des \glspl{acm}.
	
\end{e1}


\subsection{Coordination}

Une fois que la cible est désignée et approuvée, le \ja{} coordonne les points suivants \gls{cas} avec les forces au sol:

\begin{e1}
	
	\itemt{Limites de la zone de tir:}{
		Les limites de zone de tir sont les mesures de bases pour la coordination du feu.
		
		Elles sont restrictives de par le fait qu'aucun tir n'est autorisé au delà des limites.
	}
	
	\itemt{\gls{ada} alliée:}{
		Pour éviter le tir fratricide, le \ja{} annonce ``Appareil allié sur zone''.
		
		Le \ja{} coordonne avec l'\gls{ada} les \gls{cp}, \gls{ip}, la position de la cible, le nombre et le types d'appareils alliés présents, leur altitude, et leur \gls{tos}.
		
		Les \gls{spins} et l'\gls{ato} devraient inclure des \glspl{mrr} ou des corridors de retour sécurisés, ainsi les procédures associées, pour les appareils qui reviennent d'une zone de \gls{cas}.
	}
	
	\itemt{Mesures de contrôle procédural:}{
		Le contrôle procédural fournit aux pilote la direction de la cible, aligne l'appareil pour l'attaque finale ou l'egress, et fournit les séparation avec les autres tirs en cours et les menaces anti-aériennes ennemies.
		
		Le contrôle procédural inclut:
	}
	
	\begin{e2}
		
		\item{Sélection \gls{cp}/\gls{ip}\gls{bp}:}{
			Le \ja{} sélectionne les \glspl{cp}/\glspl{ip}\glspl{bp} en se basant sur les capacités de l'ennemi, l'orientation de la cible, la positions des alliés, la météo, les capacités de l'appareil \gls{cas}, et les \glspl{fscm}.
			
			Lorsque c'est possible, les \glspl{cp} et \glspl{ip} devraient être des points remarquables du terrain.
			
			L'\gls{ip} est normalement placé entre 5 et 15 miles nautiques de la cible pour les \glspl{fw}, et les \glspl{bp} entre 1 et 5 km pour les \glspl{rw}.
			
			Les appareils évoluant à plus haute altitude peuvent nécessiter un \gls{ip} plus éloigné, à 20 miles nautiques de la cible.
		}
		
		\itemt{\glsfull{kh}:} {\glsd{kh}}
		
		\begin{e3}
			\item Le Keyhole est un moyen efficace pour établir un \gls{ip} en l'absence de points de contrôle pré-définis, ou lorsque les points de contrôles déjà établis ne sont plus adaptés à la situation.
			
			Lorsqu'un appareil \gls{cas} se rapporte au \ja{} sur le \gls{cp} (=check-in), le \ja{} définit un point ``Echo''.
			
			Ce point Echo se trouve généralement sur la cible (ligne 6 de la \gls{9l}).
			
			Le \ja{} utilise ensuite ce point Echo pour définir des positions d'attente en donnant une direction et une distance à partir du point Echo.
			
			Cette distance est à considérer comme une distance \textbf{minimale} à partir du point Echo.
			
			La méthode Keyhole standard consiste à attribuer une lettre à chaque direction cardinale:
			\begin{e4}
				\item A: nord.
				\item B: est.
				\item C: sud.
				\item D: ouest.
				\item E: centre (point Echo).
			\end{e4}
			
			Le \ja{} peut également donner une radiale à partir du point Echo si les directions cardinales ne suffisent pas.
			
			\efig{khex1}{Keyhole: exemple de communication.}
			
			\begin{e4}
				\item Si la situations tactique nécessite qu'un \gls{ip} soit établie au nord de la cible, les instructions du \ja{} pourraient ressembler à:
				
				\begin{lstlisting}[caption=Keyhole: directions cardinale, label=keyholecard]
				REDWOLF ici PIRATE, pr"é"venez quand pr"ê"t "à" copier point Echo.
				PIRATE ici REDWOLF, pr"ê"t "à" copier.
				REDWOLF, coordonn"é"es point Echo suivent, 42 34.5N 043 51.2E.
				PIRATE, REDWOLF, je copie 42 34.5N 043 51.2E.
				REDWOLF, allez Alpha 8km, 200m, rappelez "é"tabli.
				...
				PIRATE, ici REDWOLF, "é"tabli Alpha 8km, 200m.
				\end{lstlisting}
				
				\item Lorsqe les directions cardinales ne suffisent pas, le \ja{} peut utiliser une radiale à partir du point Echo. Par exemple:
				
				\begin{lstlisting}[caption=Keyhole: radiale, label=keyholerad]
				REDWOLF, allez au 240 "à" 8km, 200m, rappelez "é"tabli.
				...
				PIRATE, ici REDWOLF, "é"tabli 240 "à" 8km, 200m.
				\end{lstlisting}
				
			\end{e4}
		\end{e3}
		
		
		
		\itemt{\glsfull{of}:} {\glsd{of}}
		\itemt{\glsfull{of}:} {\glsd{of}}
	\end{e2}
	\item Synchronisation
	\begin{e2}
		\item Emploi simultané
		\item Appui-feu
		\begin{e3}
			\item Marque
			\begin{e4}
				\itemt{\glsfull{idf}:} {\glsd{idf}}
				\item Feu direct
			\end{e4}
			\item \gls{aca} informelle
		\end{e3}
	\end{e2}
\end{e1}

