\section{Intégration du CAS au reste du dispositif}

L'emploi d'appareils de \gls{cas} conjointement avec le tir de surface implique une planification minutieuse combinée avec une capacité de s'adapter rapidement une situation qui évolue.\par

Des techniques de séparation sont utilisées à cette fin.

\subsection{Fire Support Coordination Measures (FSCMs)}
	Les \glspl{gc} implémentent dans leur zone des \glspl{fscm} permissives et restrictives pour accélérer le processus d'engagement des cibles; protéger les unités alliées, les populations, l'infrastructure importante, les sites religieux ou culturels; autoriser le tir Joint; établir les conditions pour préparer une opération future.
	
\begin{e1}
	
	\item La vocation principale des mesures permissives est d'accélérer l'engagement des cibles.
	
	\item La vocation principale des mesures restrictives est de protéger les forces alliées.
	
	\item Pour plus de détails concernant les \glspl{fscm}, cfr. \citetitle{jp309}.
	
	\itemt{Mesures permissives:}{}
	
	\begin{e2}
		
		\itemt{\glsfull{cfl}:} {\glsd{cfl}}
		
		\itemt{\glsfull{fscl}:} {\glsd{fscl}}
		
		\efig{fscl}{Positionnement d'une FSCL (\citetitle{jp303})}
		
		La création d'une \gls{fscl} n'implique pas une \gls{ffa} derrière la \gls{fscl}.
		
		\item BCL - spécifique à l'USMC
		
		\itemt{\glsfull{ffa}:} {\glsd{ffa}}
		
		\itemt{\glsfull{kb}:} {\glsd{kb}}
		
		\efig[\linewidth][0.4\textheight]{bluekb}{Blue Killbox, extrait du \citetitle{jp309}.}
		
		\efig[\linewidth][0.4\textheight]{purplekb}{Purple Killbox, extrait du \citetitle{jp309}.}
		
		\itemt{Statut d'un Killbox} {Une Killbox peut être ``chaude'' (HOT) ou ``froide'' (COLD):}
		
		\begin{e3}
			
			\itemt{HOT:}{L'emploi d'armement est autorisé dans la Killbox sans autre forme de coordination ou de déconfliction.}
			
			\itemt{COLD:}{L'emploi d'armement dans la Killbox n'est \emph{pas} autorisé sans coordination préalable.}
			
		\end{e3}
		
		Cfr. \citetitle{jp309}, Annexe A, pour plus d'informations à propos de la Killbox.
		
	\end{e2}
	
	\itemt{Mesures restrictives:}{}
	
	\begin{e2}
		
		\itemt{\glsfull{nfa}:} {\glsd{nfa}}
		
		\itemt{\glsfull{rfa}:} {\glsd{rfa}}
		
		\itemt{\glsfull{rfl}:} {\glsd{rfl}}
		
		\itemt{\glsfull{aca}:} {\glsd{aca}}
		
		\begin{e3}
			
			\itemt{\gls{aca} formelle}{
				L'autorité de contrôle de l'espace aérien établit l'\gls{aca} à la demande du \gls{gc}.
				
				Les \glspl{aca} formelles demandent une planification détaillée et sont intégrées dans l'\gls{aco}, l'\gls{ato} ou les \gls{spins}.
			}
			
			\efig{formalaca}{ACA formelle.}
			
			\itemt{\gls{aca} informelle}{
				Une \gls{aca} informelle est établie par le \gls{gc} au moyen d'une plan de séparation.
				
				L'\gls{aca} informelle permet d'établir une déconfliction immédiate et temporaire, et, de ce fait, ne durent généralement pas très longtemps.
				
				De par leur nature immédiate, il peut être difficile pour les autorités de contrôle de s'assurer de leur diffusion correcte.
			}
			
		\end{e3}
		
		\itemt{Séparation}{}
		
		\begin{e3}
			
			\begin{minipage}{\linewidth}
				
			\itemt{Séparation latérale}{
				La séparation latérale consiste à délimiter la zone d'engagement en deux (ou plus) parties au moyen d'une ligne.
				
				Cette ligne peut être une coordonnée MGRS, une latitude/longitude, ou un point particulier du relief.
			}
			
			\begin{lstlisting}[caption=ACA: Séparation latérale, label=acalateralsep]
				REDWOLF ici PIRATE, rester "à" l'ouest de la rivi"è"re.
			\end{lstlisting}
			
			\efig{lateralsep}{ACA: Séparation latérale.}
			\end{minipage}
			
			\begin{minipage}{\linewidth}
			\itemt{Séparation en altitude:}{
				La séparation en altitude peut être utilisée lorsqu'un appareil doit croiser la ligne de tir d'une unité alliée.
				
				La séparation en altitude en appelant ``Restez au dessus de \ldots{}'' ou ``Restez en dessous de \ldots{}''.
				
				L'altitude est données en \gls{ft} \gls{msl}, sauf si indiqué autrement.
			}
			
			\efig{altsep}{ACA: Séparation en altitude.}
			\end{minipage}
			
			\begin{minipage}{\linewidth}
			\itemt{Combinaison des séparations latérales et verticales:}{}
			
			\efig{lateralandverticalsep}{ACA: Séparation latérale en altitude.}
			\end{minipage}
			
			\itemt{Séparation horaire:}{
				La séparation horaire est celle qui demande le plus de préparation.
				
				Elle est appropriée lorsque de multiples unités alliées doivent engager la même cible.
				
				Tous les timings se basent sur les \glspl{tot}/\glspl{ttt} des appareils de \gls{cas}.
			}
			\begin{e4}
				\begin{minipage}{\linewidth}
				\itemt{\glsfull{tot}}{
					Le \gls{tot} est l'heure à laquelle les munitions tirées par l'appareil de \gls{cas} sont censées impacter la cible.
					
					C'est la méthode la plus facile à utiliser.
					
					Si le pilote ne peut pas respecter le timing, il doit en informer le \gls{tac} au plus vite.
				}
				
				\efig{tot}{ACA: Séparation horaire avec un TOT.}
				\end{minipage}
				
				\begin{minipage}{\linewidth}
				\itemt{\glsfull{ttt}}{
					Le \gls{ttt} donne un temps en minutes et en secondes entre un moment T (appelé ``HACK'') et le moment où les munitions impactent la cible.
					
					C'est une méthode précise, mais rarement utilisée.
					
					C'est le \ja{} qui donne le HACK, et les pilotes doivent en accuser réception.
				}
				
				\begin{lstlisting}[caption=Synchronisation au moyen d'un HACK, label=hacktime]
					REDWOLF ici PIRATE, Time to target 5+00, pr"ê"t... pr"ê"t... HACK.
					PIRATE ici REDWOLF, roger.
				\end{lstlisting}
				\end{minipage}
				
			\end{e4}
			
		\end{e3}
		
		\efig{fscm}{Illustration des différentes FSCMs.}
		
	\end{e2}
	\begin{minipage}{\linewidth}
	\itemt{\glsfull{acm}:} {\glsd{acm}}
	
	\begin{e3}
		\itemt{\glsfull{ca}:} {\glsd{ca}}
		\itemt{\glsfull{hidacz}:} {\glsd{hidacz}}
		\itemt{\glsfull{roz}:} {\glsd{roz}}
		\itemt{\glsfull{roa}:} {\glsd{roa}}
		\itemt{\glsfull{mrr}:} {\glsd{mrr}}
		\itemt{\glsfull{saafr}:} {\glsd{saafr}}
	\end{e3}
	\end{minipage}
	
	\efig{acm}{ACM Joint.}
	
	Cfr. \citetitle{jp352} pour plus d'informations à propos des \glspl{acm}.
	
\end{e1}


\subsection{Coordination}

Une fois que la cible est désignée et approuvée, le \ja{} coordonne les points suivants \gls{cas} avec les forces au sol:

\begin{e1}
	
	\itemt{Limites de la zone de tir:}{
		Les limites de zone de tir sont les mesures de bases pour la coordination du feu.
		
		Elles sont restrictives de par le fait qu'aucun tir n'est autorisé au delà des limites.
	}
	
	\itemt{\gls{ada} alliée:}{
		Pour éviter le tir fratricide, le \ja{} annonce ``Appareil allié sur zone''.
		
		Le \ja{} coordonne avec l'\gls{ada} les \gls{cp}, \gls{ip}, la position de la cible, le nombre et le types d'appareils alliés présents, leur altitude, et leur \gls{tos}.
		
		Les \gls{spins} et l'\gls{ato} devraient inclure des \glspl{mrr} ou des corridors de retour sécurisés, ainsi les procédures associées, pour les appareils qui reviennent d'une zone de \gls{cas}.
	}
	
	\itemt{Mesures de contrôle procédural:}{
		Le contrôle procédural fournit aux pilote la direction de la cible, aligne l'appareil pour l'attaque finale ou l'egress, et fournit les séparation avec les autres tirs en cours et les menaces anti-aériennes ennemies.
		
		Le contrôle procédural inclut:
	}
	
	\begin{e2}
		
		\item{Sélection \gls{cp}/\gls{ip}\gls{bp}:}{
			Le \ja{} sélectionne les \glspl{cp}/\glspl{ip}\glspl{bp} en se basant sur les capacités de l'ennemi, l'orientation de la cible, la positions des alliés, la météo, les capacités de l'appareil \gls{cas}, et les \glspl{fscm}.
			
			Lorsque c'est possible, les \glspl{cp} et \glspl{ip} devraient être des points remarquables du terrain.
			
			L'\gls{ip} est normalement placé entre 5 et 15 miles nautiques de la cible pour les \glspl{fw}, et les \glspl{bp} entre 1 et 5 km pour les \glspl{rw}.
			
			Les appareils évoluant à plus haute altitude peuvent nécessiter un \gls{ip} plus éloigné, à 20 miles nautiques de la cible.
		}
		
		\begin{minipage}{\linewidth}
			\itemt{\glsfull{kh}:} {\glsd{kh}}
		\end{minipage}
		
		\begin{e3}
			\item La méthode \gls{kh} standard consiste à attribuer une lettre à chaque direction cardinale:
			\begin{e4}
				\item A: nord.
				\item B: est.
				\item C: sud.
				\item D: ouest.
				\item E: centre (point Echo).
			\end{e4}
			
			Le \ja{} peut également donner une radiale à partir du point Echo si les directions cardinales ne suffisent pas.
			
			\efig{khex1}{Keyhole: exemple de communication.}
			
			\begin{e4}
				
				\begin{minipage}{\linewidth}
				\item Si la situations tactique nécessite qu'un \gls{ip} soit établie au nord de la cible, les instructions du \ja{} pourraient ressembler à:
				
				\begin{lstlisting}[caption=Keyhole: directions cardinale, label=keyholecard]
				REDWOLF ici PIRATE, pr"é"venez quand pr"ê"t "à" copier point Echo.
				PIRATE ici REDWOLF, pr"ê"t "à" copier.
				REDWOLF, coordonn"é"es point Echo suivent, 42 34.5N 043 51.2E.
				PIRATE, REDWOLF, je copie 42 34.5N 043 51.2E.
				REDWOLF, allez Alpha 8km, 200m, rappelez "é"tabli.
				...
				PIRATE, ici REDWOLF, "é"tabli Alpha 8km, 200m.
				\end{lstlisting}
				\end{minipage}
				
				\begin{minipage}{\linewidth}
				\item Lorsqe les directions cardinales ne suffisent pas, le \ja{} peut utiliser une radiale à partir du point Echo. Par exemple:
				
				\begin{lstlisting}[caption=Keyhole: radiale, label=keyholerad]
				REDWOLF, allez au 240 "à" 8km, 200m, rappelez "é"tabli.
				...
				PIRATE, ici REDWOLF, "é"tabli 240 "à" 8km, 200m.
				\end{lstlisting}
				\end{minipage}
				
			\end{e4}
			
			\efig{keyhole}{Keyhole: illustration}
			
			\item Le \gls{kh} permet une flexibilité illimitée pour établir des \glspl{ip}, et d'éviter d'avoir à planifier des \glspl{ip} qu'on utilisera probablement jamais partout dans la zone opérationnelle.
			
			Cependant, en utilisant le \gls{kh}, le \ja{} n'aura probablement pas de point de référence visuel ou géographique marquant via lequel orienter les appareils.
			
		\end{e3}		
		
		\itemt{Overhead:}{L'Overhead est une attaque à partir de la verticale de la zone cible.}
		\begin{e2}
			\item Pour les attaques au moyen de \glspl{pgm}, une distance accrue sera nécessaire et devrait être spécifier en ligne 1-3.
		
			Les attaques aux \glspl{pgm} ne permettent pas l'Overhead.
		
			\item Si l'appareil attaque depuis la verticale, les lignes 1-3 peuvent être ``De la verticale'', ou ``Lignes 1 à 3 pas d'application''.
		\end{e2}
		
		\begin{minipage}{\linewidth}
			\itemt{\glsfull{of}:} {\glsd{of}}
		\end{minipage}
		
		\efig{offset}{Décalage (offset).}
		
		\begin{minipage}{\linewidth}
			\itemt{\glsfull{fah}:} {\glsd{fah}}\label[para]{tofah}
		\end{minipage}
		
		\begin{e3}
			\item Le \ja{} doit évaluer les avantages d'imposer un \gls{fah} par rapport aux inconvénients et ne l'utiliser que si c'est nécessaire. Un minimum de contraintes doivent être imposées au pilote.
			
			\item Toute géométrie d'attaque passée lors du CAS-brief est considérée comme une restriction, et doit être répétée par le pilote lors du read-back.
			\begin{minipage}{\linewidth}
			\item Un passage de \gls{fah} peut ressembler à:
			\begin{lstlisting}[caption=FAH: cap magnétique, label=fah1]
			Cap d'attaque final 230.
			\end{lstlisting}
			\begin{lstlisting}[caption=FAH: cap magnétique avec cône, label=fah2]
			Cap d'attaque final 240-300.
			\end{lstlisting}
			\begin{lstlisting}[caption=FAH: cap magnétique avec une fourchette, label=fah3]
			Cap d'attaque final 270 plus ou moins 30 degr"é"s.
			\end{lstlisting}
			\begin{lstlisting}[caption=FAH: direction cardinale, label=fah4]
			Cap d'attaque final du nord-est au sud-ouest.
			\end{lstlisting}
			\begin{lstlisting}[caption=FAH: référence géographique, label=fah5]
			Effectuez toutes les attaques parall"è"lement "à" la route.
			\end{lstlisting}
			\end{minipage}
		\end{e3}
		
	\end{e2}
	\itemt{Synchronisation}{}
	\begin{e2}
		\itemt{Emploi simultané}{
			Le but est de protéger les appareils alliés tout en limitant au maximum le temps pendant lequel les autres armes ne peuvent pas tirer de peur de toucher un appareil en attaque.
		}
		\itemt{Temps de référence commun}{
			Pour permettre la synchronisation de tous les éléments, il est impératif que tous les participants utilisent la même méthode de timing (\gls{tot} ou \gls{ttt}).
		}
		
		\itemt{Appui-feu pour le \gls{cas}:}{
			Il y a deux formes principales d'appui-feu pour le \gls{cas}: le marquage et le \gls{sead}.
		}
		\begin{e3}
			\itemt{Marque}{
				Une marque devrait être fournie aux appareils \gls{cas} chaque fois que c'est possible.
				
				Il faut prévoir le temps nécessaire au pilote pour acquérir la marque.
				
				le \ja{} doit prévoir un système de marquage redondant au cas ou la méthode prévue en premier lieu ne fonctionne pas.
				
				La marque peut être fournie par \gls{idf} ou par un \gls{faca}.
				
				Si l'usage d'une marque visuelle n'est pas possible, on utilisera une marque ``vocale'', qu'on appelle le ``talk-on''. Le \ja{} va simplement diriger l'attention du pilote vers la cible en discutant avec lui. Le \gls{jtac} peut marquer sa propre position pour aider le talk-on (miroir, fumée, \ldots{}), en faisant attention à ne pas révéler les positions alliées à l'ennemi.
				
			}
			\begin{e4}
				
				\begin{minipage}{\linewidth}
					
				\item Le marquage implique l'emploi d'une brevity particulière:
				
				\efig{markbrevity}{Brevity pour la marque.}
				
				\end{minipage}
				
				\itemt{Marque par \gls{idf}:} {
					L'artillerie, les mortiers et le tir depuis une unité navale sont des moyens efficaces de marquer la cible.
					
					Lorsqu'il y a déjà beaucoup d'activité sur le champ de bataille, il faut faire attention à ce que la marque ne soit pas confondue avec une autre activité au sol.
					
					Le timing pour la marque est très important. Le \ja{} doit prendre en compte le temps nécessaire pour que la marque soit effective, et le temps nécessaire au pilote pour commencer à scanner la zone.
					
					La marque est la plus efficace lorsqu'elle tombe à moins de 100m de la cible, mais suffit généralement à guider les appareils \gls{cas} jusqu'à 300m de la cible.
				}
				\itemt{Marque par feu direct:}{
					Bien que cette méthode soit plus facile à mettre en place que l'\gls{idf}, les marques par feu direct sont généralement beaucoup moins visibles depuis les airs.
				}
				\itemt{Marque par \gls{ltd}}{
					Pour les appareils équipés d'un \gls{lst}, la marque laser est très efficace.
					
					Il faut cependant faire attention à ce que le pilote ne confonde pas la source du pointeur et la zone pointée.
					
					Lors de l'emploi d'une marque laser, le call-sign de la station qui effectue le pointage et le code laser utilisé seront donnés.
					
					\important{Le \ja{} doit import un \gls{fah} pour s'assurer que l'appareil acquiert la zone pointée et non pas la source du pointage.}
					
					Le pilote préviendra 10 secondes avant de demander l'activation du laser.
					
					\begin{minipage}{\linewidth}
						
						Le marquage laser implique l'emploi d'une brevity particulière:
						
						\efig{brevitylaser}{Brevity pour la marque laser.}
						
					\end{minipage}
				}
				\itemt{Marque par \gls{faca}}{
					Certains appareils \gls{faca} peuvent marquer la cible au moyen de roquettes au phosphore, \gls{ltd}, pointeurs \gls{ir}, ou munitions traçantes.
				}
				\itemt{Marque par pointeur \gls{ir}}{
					Le \ja{} peut utiliser un pointeur \gls{ir} pour marquer la cible de nuit, à condition que les pilotes soient équipés de \glspl{nvg}.
					
					Contrairement au pointeurs lasers, les pointeurs \gls{ir} ne peuvent pas être utilisés pour guider une munition.
					
					L'utilisation de pointeurs \gls{ir} doit se faire avec prudence, car ils peuvent révéler la position du \gls{jtac} à l'ennemi.
					
					Le pointage \gls{ir} devrait idéalement commencer 20 à 30 secondes avant le \gls{tos}/\gls{ttt}, ou lorsque le pilote le demande.
					
					\begin{minipage}{\linewidth}
						
						Le marquage \gls{ir} implique l'emploi d'une brevity particulière:
						
						\efig{brevityir}{Brevity pour la marque infrarouge.}
						
					\end{minipage}
				}
				\itemt{Marques combinées}{
					Le \ja{} peut décider d'employer plusieurs marques de types différents.
				}
				\begin{minipage}{\linewidth}
					
					\efig{laserwarning}{Avertissement laser et infrarouge.}
					
				\end{minipage}
				\itemt{Marque des alliés}{
					Le marquage des unités alliées est la pire méthode pour obtenir un TALLY.
					
					Cette méthode peut être déroutante et ne doit être utilisée, avec prudence, que si aucune autre méthode n'est possible.
				}
			\end{e4}
			\itemt{\gls{sead}}{}
			\begin{e4}
				\itemt{L'objectif principal du \gls{sead} est de permettre aux appareils alliés d'opérer dans un espace aérien défendu par un système anti-aérien ennemi.}{
					Cela inclut les routes d'ingree/egress.
					
					Les missions \gls{sead} ne garantissent pas l'immunité.
					
					Le \ja{} doit évaluer les différents profils de missions possibles pour minimiser l'exposition des appareils alliés aux défenses ennemies connues ou potentielles.
					
					S'il est impossible d'éviter les défenses ennemies, la vulnérabilité des appareils \gls{cas} doit être mise en balance avec le risque encourus par les appareils \gls{sead} avant de décider si une mission \gls{sead} s'impose.
				}
				\itemt{Coordination}{
					Le \gls{sead} depuis une plateforme aéroportée doit être coordonné et intégré au plan.
					
					Avant de demander un \gls{cas} nécessitant du \gls{sead}, il faut chercher à accomplir la neutralisation des défenses anti-aériennes ennemies par d'autres moyens (\gls{idf}).
					
					Le \gls{sead} est le plus efficace contre les \glspl{sam} et \gls{aaa}, et le moins efficace contre les \glspl{manpad} ou les défenses très mobiles.
				}
			\end{e4}
		\end{e3}
	\end{e2}
\end{e1}

