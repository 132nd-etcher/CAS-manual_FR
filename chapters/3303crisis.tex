\section{Planification de crise}

La planification de crise consiste à développer in \gls{oplan} et un \gls{opord} dans un temps limité en réponse à une situation de crise imminente.

Du point de vue du \gls{cas}, ce processus est plus proche du niveau tactique que stratégique.

Cfr. \citetitle{jp50} pour plus d'informations concernant la planification de crise.