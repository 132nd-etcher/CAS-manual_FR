\section{Introduction - Résumé, organisation et fondamentaux}

\begin{e1}
	\itemt{Le \acrfull{cas} est une action aérienne effectuée par les \glsfullpl{fw} et des \glsfullpl{rw} contre des cibles hostiles qui se trouvent à proximité de forces amies et qui nécessite une intégration minutieuse des missions aériennes avec le feu et le mouvement des forces amies.}{
	Le \gls{cas} demande une planification en détail, de la coordination et de l'entraînement.
	
	Le \gls{gc} soutenu par le \gls{cas} établit les priorités, les effets et le timing du \gls{cas} dans les limites de sa zone de contrôle. Le \gls{cas} donne la capacité au \gls{gc} d'employer la force aérienne pour détruire, empêcher ou neutraliser les forces ennemies et permet en conséquence le mouvement, la manoeuvre et le contrôle de territoire, de population et de zones maritimes.
	
	Le \gls{cas} est planifié et exécuté de manière à soutenir les unités au sol tactiques. Il est étroitement intégré au niveau tactique. L'allocation des appareils qui effectuent le \gls{cas} se fait au niveau opérationnel (stratégique). La planification du \gls{cas} vie à fournir un appui-feu précis et rapide aux forces amies au contact avec l'ennemi.
	
	Chaque service s'organise, s'entraîne et s'équipe pour employer le \gls{cas} selon son rôle dans la coalition. Un grand nombre d'appareils différents sont capables d'effecteur le \gls{cas}. Le \gls{jfc} et son staff ont la responsabilité d'intégrer ces capacités \gls{cas} dans leur \gls{conops}.}
	
	\itemt{\glsfull{tac}}{Le \glsfull{tac} est l'autorité qui contrôle la manoeuvre des appareils de \gls{cas} et qui autorise le tir de munitions.}
	
	\itemt{\glsfull{tgo}}{La guidance terminale diffère \gls{tac}. Les \glspl{tgo} sont les actions qui fournissent des informations additionnelles quant à la position de la cible, par des moyens électroniques, mécaniques, visuels ou vocaux.
	
	A moins qu'il aie la qualification de \glsfull{jtac}, le personnel qui effectue la \gls{tgo} n'a pas l'autorité pour contrôler la manoeuvre des appareils de \gls{cas}, ou pour autoriser le tir de munitions.}
	
	\itemt{Utilité sur le champ de bataille}{
	
	Le \gls{cas} donne aux \glspl{gc} un appui-feu flexible et réactif.
	
	En utilisant le \gls{cas}, les \glspl{gc} peuvent mieux tirer parti des opportunités qui se présentent sur le champ de bataille, en concentrant la puissance de feu pour pousser un avantage tactique, ou pour réduire les risques tactiques et opérationnels. Le mobilité de la vitesse des forces aériennes donnent aux \glspl{gc} un moyen de frapper l'ennemi repidement et de manière inattendue.}
	
	\itemt{Intégration du \gls{cas}}{
	
	Pour des opérations Joint, l'intégration du \gls{cas} commence au niveau opérationnel, lors du cycle de planification.
	
	Le \glsfull{jfacc} consulte les commandants des différentes composantes, puis donne ses recommandations au \gls{jfc}, qui établit l'allocation des appareils en sorties pour un certaine période de temps. Le \gls{jfc} décide en fonction des cibles prioritaires, des avis qui lui sont données, et des objectifs de l'opération.}
	
	\itemt{Conditions pour que le \gls{cas} soit efficace}{}
	\begin{itemize}[label={*}]
		\item Personnel bien entraîné.
		\item Planification efficace.
		\item Structure \gls{cc} efficace.
		\item Supériorité aérienne (tout spécialement la neutralisation des défenses anti-aériennes ennemies).
		\item Un marquage et/ou une acquisition des cibles corrects.
		\item De l'armement approprié.
		\item Bien que moins importantes, des conditions environnementales clémentes améliorent également l'efficacité du \gls{cas}.
	\end{itemize}
	
	\itemt{Minimiser le tir fratricide}{
	
	Tous les participants au \gls{cas} sont responsables pour la planification et l'exécution efficaces et sécurisées du \gls{cas}. Chaque participant doit faire tout son possible pour identifier les unités amies, les forces ennemies, et les civils avant de cibler, d'autoriser le tir, ou de tirer. Le \glsfull{cid} est le processus par lequel on obtient une classification des objets détectés dans un environnement de combat suffisamment satisfaisante que pour autoriser un engagement.
	
	La classification par le \gls{cid}, en parallèle avec les \glsfullpl{roe}, permettent de prendre d'autoriser ou de refuser l'engagement au moyen d'arme létales ou non-létales pour accomplir un objectif militaire.}
	
	\itemt{Minimiser les pertes civiles}{
	
	Les lois de la guerre obligent tous les commandants à prendre toutes les mesures en leur pouvoir pour minimiser les pertes civils et les dommages collatéraux, en restant dans le cadre des objectifs de la mission et en assurant la sécurité de la force dont ils sont responsables.}
	
	\itemt{Commandement et contrôle}{
	Le \gls{cas} requiert une structure \gls{cc} intégrée, flexible et réactive pour traiter les exigences du \gls{cas}, ainsi qu'une architecture de communications sécurisée et fiable pour exercer le contrôle.
	
	Normalement, le \gls{jfc} exerce son \glsfull{opcon} par le truchement des commandants de composante. La plupart des unités de \gls{cas} seront allouées au et taskées apr le \gls{jfacc}.}
	
	\itemt{Communications}{
	
	Les missions de \gls{cas} demandent un grand degré de contrôle exercé au moyen de communications efficaces. Les communications doivent être souples et réactives pour maintenir la liaison entre les appareils et les unités au sol, et ainsi réduire le risque de tir fratricide.}
	
	\itemt{Renseignements}{
	
	L'espionnage, la surveillance et la reconnaissance (\glsfull{isr}) offrent leur support au \gls{cas} depuis le début du processus de planification jusque lors de son exécution.}
	
	\itemt{Le \gls{cas} lors de la prise de décision}{
	
	Le \gls{cas} est pris en compte par les différents commandants lors du processus de décision pour la création d'un plan d'appui-feu.
	
	Les différentes parties participent activement à ce processus en fournissant leurs inputs à propos du \gls{cas}.
	
	La phase de planification se termine lors de la publication du plan aux unités subordonnées.}
	
	\itemt{Planification du \gls{cas}}{
	
	Le \gls{cas} peut soutenir trois grands types d'opérations:}
	\begin{e2}
		\itemt{Opérations de profondeur}{
		
		Ces opérations ont lieu loin de la ligne de front, pour soutenir des forces spéciales ou conventionnelles en territoire ennemi. Ces opérations peuvent demander une coordination supplémentaires avec les forces aériennes d'interdiction (cfr. \gls{das}).}
		
		\itemt{Opérations de combat rapproché}{
		
		Le commandant assigne généralement le gros du \gls{cas} à l'effort de guerre principal (=la ligne de front). La vitesse, la portée et la puissance du \gls{cas} en font un atout majeur pour exploiter une avancée amie, interrompre une poussée ennemie, et attaquer un ennemi en retraite.}
		
		\itemt{Opérations de sécurité}{
		
		Le \gls{cas} est efficace pour repousser un ennemi qui approche. Cependant, lors d'une opération de sécurité, le danger de tir fratricide est élevé, à cause du grand nombre d'unités et d'activités amies dans la zone.}
	\end{e2}
	
	\itemt{Considérations particulières pour la planification}{}
	
	\begin{e2}
		\itemt{Ennemi}{
		
		Les planificateurs doivent anticiper la capacité de l'ennemi à impacter la mission, et l'influence que pourraient avoir ses actions sur la tactique alliée. Lorsque la menace augmente, le briefing des pilotes prend de plus en plus d'importance. La possibilité que la menace change durant la mission rend l'importance des communications et de la coordination avec les forces au sol capitales.}
		
		\itemt{Troupes}{
		
		Les planificateurs doivent prendre en compte le capital \gls{cc}, \gls{isr}, \gls{ew}, ainsi que les appareils \gls{cas} disponibles.
		
		Un appareil \gls{cc} allié peut faciliter les choses, mais il faut tenir à l'esprit que chaque plateforme à ses capacités et limitations propres.
		
		Toutes les sources d'\gls{isr} doivent être utilisées.}
	
		\itemt{Terrain}{
	
		Le terrain peut affecter les communications et la \gls{los}. Les \glspl{rw} sont extrêmement vulnérables.}
		
		\itemt{Timing}{
		
		Les demandes de \gls{cas} doivent être suffisamment à l'avance que pour pouvoir être intégrées au processus de planification. Toute demande reçue trop tard sera traitée comme une demande de \gls{cas} immédiate.}
	\end{e2}
	
	\itemt{Intégration du \gls{cas}}{}
	
	\begin{e2}
		\itemt{\glsfullpl{fscm}}{
		
		Les \glspl{gc} établiront des \glspl{fscm} dans leur zone de contrôle. Ces \glspl{fscm} sont des mesures qui spécifient clairement le type de restrictions de feu qui s'applique dans telle ou telle zone.}
		
		\itemt{\glsfullpl{acm}}{
		
		Les \glspl{acm} sont des mesures qui visent à sécuriser les appareils alliés dans des parties d'espace aérien délimitées.}
		
		\itemt{Coordination}{
		
		Une fois la cible du \gls{cas} définie, le \ja{} et le centre des opérations coordonnent le \gls{cas} avec les forces au sol alliées.}
	\end{e2}
	
	\itemt{Tactiques employées par les appareils de \gls{cas}}{
	
	Les tactiques sont en constante évolution; la décision d'utiliser telle ou telle tactique appartient, au final, au pilote. Celui-ci doit cependant intégrer sa tactiques aux contraintes et restrictions qui lui sont imposées par le ja{}.}
	
	\itemt{\glsfull{faca}}{
	
	Le \gls{faca} peut servir de contrôleur supplémentaire pour le \gls{tacp}/\gls{jtac}, soutenir un élément qui ne dispose pas de \gls{tacp}/\gls{jtac}, ou suppléer les capacités du \gls{tacp}/\gls{jtac}.}
	
	\itemt{\glsfull{taca}{
	
	Le \gls{taca} est une extension du centre opérationnel. Son autorité peut varier du simple relais radio à pouvoir autoriser le décollage, le délais ou la diversion des appareils alliés.}
	
	\itemt{Demande de \gls{cas}}{
	
	Les \glspl{req} sont émises par les \glspl{gc}. Il y a deux types de demandes:}
	\begin{e2}
		\itemt{Demande planifiée}{
		
		Les missions de \gls{cas} planifiées sont remplies par des appareils spécifiquement tasqués pour cette mission, ou par des appareils on-call.}
		
		\itemt{Demande immédiate}{
		
		Les missions de \gls{cas} immédiates sont remplies par des appareils on-call.}
	\end{e2}
	
	\itemt{Préparation}{}
	
	La préparation à la mission englobe les points suivants:
	
	\begin{e2}
		\item Répétitions
		\item Préparations avant le combat
		\item Communications
	\end{2}
	
	\itemt{Exécution}
	
	Bien que chaque mission soit unique, il est important que le \ja{} et les pilotes soient familiers avec le déroulement global d'une opération \gls{cas}.	
	
	\begin{e2}
		\itemt{\gls{jtac}}{
		Une fois que le \gls{gc} a désigné la cible, le \gls{jtac} doit effectuer les suivantes:}
		\begin{e3}
		\itemt{Rassembler les données à propos de la cible}{Données minimales:}
			\begin{e4}
				\item Élévation
				\item Description
				\item Position
				\item Unités amies aux alentours
			\end{e4}
		\itemt{Émettre la \gls{req}}{
		Une fois les intentions du \gls{gc} claires et la position de la cible grossièrement établie, le \jtac{} doit envoyer au plus vite la \gls{req}, de manière à donner le temps aux appareils de \gls{cas} d'arriver sur zone.}
		\itemt{Développer le game-plan}{
		Au minimum, le game-plan contiendra le type de contrôle et la méthode d'attaque.
		
		De plus, il pourra contenir:}
		\begin{e4}
			\item Intentions du \gls{gc}.
			\item Effet désiré.
			\item Armement souhaité.
			\item Intervalle entre les appareils.
		\end{e4}
		\itemt{Déterminer/coordonner la marque/aide à la corrélation}{
		La marque et la corrélation dépendront du type d'attaque et du type d'appareil qui attaque.}
		\itemt{Déterminer la géométrie d'attaque}{
		Il y a de nombreux facteurs à prendre en compte lors de l'établissement de la géométrie d'attaque.}
		\itemt{Déterminer le plan \gls{sead}}{
		Si l'évitement de la menace n'est pas possible, il faut établir un plan \gls{sead}. Ce plan peut être extrêmement complexe en fonction de la menace présente.}
		\end{e3}
		\itemt{Exécution du \gls{cas} sans \ja{}}{
		Si un \gls{gc} a besoin de \gls{cas} mais n'est pas à même de fournir un \gls{jtac} et qu'un \gls{faca} n'est pas présent, le pilote aura la responsabilité supplémentaire de minimiser le tir ami et les dommages collatéraux.}
		\itemt{Intégration du \glsfull{faca}}{}
		\begin{e3}
			\itemt{Check-in}{
			Le \gls{faca} est une extension du \gls{tacp}, et doit indiquer ``Qualifié FAC(A)µµ lors du check-in, indiquant ainsi au \gls{jtac} les capacités supplémentaires de l'appareil.}
			\itemt{Devoirs et responsabilités}{Le \gls{jtac} et le \gls{faca} doivent rapidement se partager les devoirs et responsabilités pour le contrôle.
			
			Les trois objectifs du \gls{faca} sont: accomplir les intentions du \gls{gc}, maximiser et intégrer l'appui-feu, et minimiser le tir fratricide.}
		\end{e3
	\end{e2}

\end{e1}
