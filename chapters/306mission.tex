\newpage
\section{Mission}

Le \gls{cas} est intégré au reste du dispositif de la coalition en opérations défensives et offensives, pour détruire, neutraliser, interdire, retarder ou empêcher le mouvement de l'ennemi.
\e
	\item Le \gls{cas} peut servir pour renforcer les opérations annexes, soutenir l'effort majeur ou établir les zones de sécurité
	\ee
		\itemt{Opérations annexes}{
		Bien que ce ne soit pas vocation principale, le \gls{cas} peut être employé pour soutenir une opération loin en territoire ennemie (infiltration, exfiltration, opérations spéciales), ou pour une opération ponctuelle}
		\itemt{Opérations de combat rapproché}{
		Le plus souvent, le \gls{cas} servira à renforcer l'effort de guerre principal. Les appareils de \gls{cas} ajoutent leur force aux forces au sol pour soutenir les \glspl{gc}. La vitesse et la portée du \gls{cas} en font un élément idéal pour exploiter les avancées amies, contrer les manoeuvres adverses ou poursuivre un ennemi en fuite.}
		\itemt{Opérations de sécurité}{
		Le \gls{cas} est efficace pour empêcher la pénétration de l'ennemi. Le temps de réponse et la puissance de feu du \gls{cas} peut augmenter de beaucoup la force des unités au sol.}
	\ed
	\item Le \gls{cas} peut servir pour les opérations offensives, défensives et de stabilité
	\ee
		\itemt{Le \gls{cas} en support de l'offense}{
		\eee
			\itemt{Mouvement vers le contact}{
			Le \gls{cas} peut servir à appuyer les forces au sol qui font mouvement. Une fois le contact avec l'ennemi établi, le \gls{cas} peut les submerger et accélérer le déploiement des forces alliées. \textbf{Pendant la planification de l'intégration du \gls{cas} à l'avancée alliée, il est recommandé de déployer les appareils tout le long de l'axe de progression.}}
			\itemt{Attaque}{
			Les \gls{gc} peuvent utilser le \gls{cas} pour attaquer directement l'ennemi. Le \gls{cas} peut détruire les unités ou capacités ennemies importantes avant que l'ennemi ne puisse établir de défense. Le \gls{cas} peut également servir à renforcer la puissance de feu lors d'une attaque, ou pour isoler une force ennemie sur le champ de bataille et la forcer à prendre une position défensive.}
			\itemt{Exploitation}{
			L'exploitation est une tactique offensive utilisée après une attaque réussie, et sert à déstabiliser l'ennemi, en coupant les voies d'accès, en détruisant les forces qui font retraite, et pour frapper les cibles d'opportunité lorsque la cohésion de l'ennemi diminue.}				\itemt{Poursuite}{
			En poursuite, le \gls{gc} attaque l'efficacité de l'ennemi en fuite alors qu'il est démoralisé et que sa cohésion et sont contrôle sont faibles. Puisque l'objectif de la poursuite est la destruction de l'ennemi, le \gls{cas} peut maintenir une pression directe et constante sur l'ennemi pour empêcher qu'il ne se réorganise ou ne se reconstitue.}
		\ed}
		\item Le \gls{cas} en support des opérations défensives alliées
		\eee
			\itemt{Support de la manoeuvre}{
			Le \gls{cas} peut s'ajouter à la puissance de feu des unités au sol en tant que membre d'une force combinée}
			\itemt{Support du mouvement}{
			Le \gls{cas} peut soutenir les éléments alliés lors de leurs mouvements entre deux positions. Les \gls{gc} peuvent utiliser le \gls{cas} pour protéger le front, les flancs ou l'arrière de la formation en mouvement.}
			\itemt{Refoulement de l'ennemi}{
			Le \gls{cas} peut servir à refouler une force ennemie qui aurait franchi ou pénétré les défenses alliées}
		\ed
		\itemt{Le \gls{cas} en opération de stabilité}{
		L'emploi du \gls{cas} en opération de stabilité est foncièrement différent de l'emploi du \gls{cas} en opération de combat. Comme l'objectif d'une opération de stabilité est d'établir la sécurité civile, rétablir les services, et restaurer et protéger les infrastructures, le \gls{cas} en opération de stabilité sera généralement restreint en terme d'échelle et des \glspl{roe} strictes seront souvent d'application. L'utilisation de \glspl{pgm} sera souvent préconisé par le \gls{jfc} lors du support de \gls{gc} en environnement urbain, de manière à limiter les dommages collatéraux.
		}
	\ed
\ed