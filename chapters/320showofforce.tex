\section{Show of Force}

Le \gls{gc} peut décider que la meilleur \gls{coa} pour l'instant consiste à faire une démonstration de force plutôt que d'utiliser la force létale. Une démonstration de force est une opération visant à afficher la résolution des troupes alliées en les montrant ouvertement à l'ennemi, en espérant désamorcer ainsi l'escalade de la violence.

Les \glspl{jtac} doivent se rappeler que le Show of Force est une forme non-létale d'utilisation de la force aérienne, et que les restrictions ne s'appliquent pas. Bien que non-létal, un Show of Force doit avoir une cible et des effets souhaités, et respecter les \glspl{roe} du théâtre d'opération ainsi que les \gls{spins}.

\begin{e1}
	\item Raisons pour effectuer un Show of Force
	\begin{e2}
		\item L'appareil n'a plus de munitions, ou n'a pas le bon type de munitions
		\item Les forces ennemies sont trop proches des forces alliées que pour pouvoir les engager
		\item Les forces alliées ou le pilote ne parviennent pas a établir la position des forces ennemies de manière satisfaisante
		\item Une unité alliée avec laquelle le \gls{jtac} n'est pas en contact radio rencontre l'ennemi ou se trouve en situation d'escalade de la violence, et le \gls{jtac} utilise le Show of Force pour la rassurer et lui montrer que le support aérien est présent et disponible.
		\item Le \gls{gc} souhaite disperser un rassemblement de civils
		\item Du personnel inconnu montre des intentions hostiles et le \gls{gc} ne souhaite pas (encore) recourir à une réponse armée. \gls{jtac} peut utiliser le Show of Force pour aider à déterminer les intentions ou à disperser ce personnel inconnu
	\end{e2}
	\remark{Le Show of Force implique généralement que l'appareil quitte une zone sécurisée et s'expose à une menace potentielle.}
\end{e1}
