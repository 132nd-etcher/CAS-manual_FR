\newpage
\section{Concepts essentiels pour un CAS efficace}

\note{Cette section et ses sous-sections sont fortement résumées pour s'inscrire dans le contexte de DCS}

\subsection{Battle Tracking}

Le Battle Tracking consiste à se maintenir au courant de l'évolution de la situation.

\e
	\item Zones dans lesquelles le vol est interdit ou soumis à des restrictions
	\item Zones dans lesquelles le feu est interdit ou soumis à des restrictions
	\item Informations quant aux unités alliées
	\item Artillerie
	\item Positions de l'ennemi
	\item Cibles
	\item \gls{frago}, update de l'\gls{ato}
	\item Updates du \gls{complan}
\ed

\subsection{Relation entre l'élément soutenu et l'élément qui apporte le soutien}

\note{Cette sous-sections a été adaptée pour s'inscrire dans le contexte de DCS}

Dans un engagement \gls{cas}, l'unité supportée est le \gls{gc}. L'élément de support est constitué du \ja{} et des pilotes. Le \ja{} est le représentant du \gls{gc}, et les informations transmises par lui doivent être considérées comme émanant du \gls{gc}. L'élément de support doivent fournir le plus d'informations possible au \gls{gc}. \textbf{Une fois que le \gls{gc} est en possession de toutes les informations, et que l'élément de support ont acquis la cible correcte, la responsabilité d'employer de l'armement dans sa zone de contrôle lui incombe, fussent cet armement tiré par l'élément de support}.

\subsection{Ciblage et TLE}

\e
	\item La \gls{tle} est différence entre la position réelle et la position estimée de la cible. Cette différence doit être adaptée au scénario.

	\ee
		\item Une zone urbaine contenant des unités alliées demandera une faible \gls{tle}
		\item Sur un champ de bataille plus conventionnel, une plus grande \gls{tle} sera acceptable
	\ed
	
	\item La \gls{tle} doit être communiqué s'il est attendu qu'elle aie un impact significatif sur la mission. Autrement, elle est optionnelle.

	\item Pour simplifier les communications, la \gls{tle} est exprimée en 6 catégories.

	\efig{tle}{Target Location Error}
	
	\note{Ignorez les "CE", "VE" et "SE" sur l'image ci-dessus}
	
	\item S'il souhaite fournir la \gls{tle}, le \ja{} le fera lors du game-plan, avant le CAS brief.
\ed

\subsection{Dommages collatéraux}

Tous les planificateurs de \gls{cas} s'emploient, dans la limite des contraintes imposées par la mission, le timing, et la protection des forces amies, à minimiser les dommages collatéraux.

Responsabilités du \ja{}:

\e
	\item Comprendre les causes majeures de dommage collatéral:
	\ee
		\item Mauvaise identification de la cible hostile ou de l'élément civil
		\eee
			\item \gls{cid}
			\item Attention particulière si une tierce partie est utilisée pour l'acquisition
		\ed
		\item Armement inapproprié
		\eee
			\item Armement de moindre puissance
			\item Armement plus précis
			\item Éviter les armes à fragmentation
			\item Utiliser un décalage latéral pour mitiger les effets de l'armement
		\ed
		\item Armement défectueux
		\eee
			\item Axe d'attaque dégagé en cas de tir raté
			\item Orienter les fragments des armes à fragmentation
		\ed
	\ed
\ed

\important{Parfois, certaines cibles seront d'une telle importance stratégique que les dommages collatéraux seront sciemment inclus comme une nécessité opérationnelle dans la planification du \gls{cas}}

\newpage
\subsection{Autorisation de tir}

Le \gls{gc} délègue l'autorisation d'employer de l'armement dans sa zone au \ja{}. Cette autorisation permet au \ja{} d'émettre les appels suivants:

\efig{clearancecalls}{Appels pour l'autorisation de tir}

\e
	\item \acrfull{abort} 
	\item \acrfull{clearedhot}
	\item \acrfull{continue}
	\item \acrfull{continuedry}
	\item \acrfull{clearedeng}
\ed

\subsection{Annulation après le tir}

Certaines munitions guidées offrent la possibilité d'effectuer un \gls{pla}. Étant donné le temps de vol généralement très court des armes guidées, cette procédure est à effectuer en dernier recours et le plus tôt possible après le lancement.

\e
	\item L'impossibilité d'effectuer le \gls{pla} doit être signalée par le pilote au moyen de l'appel "UNABLE"
	\item Procédure pour le \gls{pla}:
	\ee
		\item Un CAS brief standard est transmis
		\item Le point de \gls{pla} sera communiqué dans les restrictions, ainsi que les circonstances dans lesquelles le \gls{pla} sera d'application
		\item Le pilote effectue un read-back incluant le \gls{pla}
		\item Si nécessaire, l'appel \gls{abort} sera transmis pour initier le \gls{pla}
	\ed
\ed

