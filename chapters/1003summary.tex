%\thispagestyle{default}

% \section*{Préface}
% \phantomsection
% \addcontentsline{toc}{section}{\protect\numberline{}Préface}
\invisibleunnumberedchapter{Résumé}%

%\setlength{\parindent}{10em}
\begin{imini}
\subsection*{Cadre}

\vfil

Ce document décrit les procédures \acrfull{cas} Joint en vigueur telles que décrites dans le \jp{}.

\vfill

\subsection*{Trivia}

\vfil

De par sa nature inter-armes, le Close Air Support (CAS) nécessite un ensemble de procédures reconnues et utilisées par toutes les parties; c'est pourquoi les procédures Joint décrites dans le \jp{} sont utilisées.

\vfil

Le \jp{} est un publication Joint qui établit le cadre et les procédures pour la conduite du Close Air Support dans une coalition inter-armes ou internationale. Ce document est utilisé, entre autres, par les différentes composantes de l'armée américaine et par l'Organisation du Traité Atlantique Nord (OTAN).

\vfil

Les procédures du \jp{} sont implémentées de manière libérale; certaines parties ou spécificités sont omises, soit parce qu'elles ne s'appliquent pas à notre niveau, soit parce que la valeur tactique ou immersive ajoutée ne contrebalance pas suffisamment la complexité intrinsèque à leur mise en place.

\vfil

Il est entendu que les procédures implémentées sont celles qui s'appliquent tout particulièrement aux appareils à voilure tournante, et que les procédures propres aux appareils à voilure fixe sont implicitement omises ou survolées brièvement.

\vfil

\textbf{Aucune modification n'a été apportée aux procédures décrites dans le \jp{}, et toute interprétation erronée serait involontaire.}
%\setlength{\parindent}{\defaultparindent}
\end{imini}