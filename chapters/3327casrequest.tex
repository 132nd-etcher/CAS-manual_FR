\section{Requête CAS}
\label{casrequestsection}

Il existe deux types de \gls{req}: planifié et immédiat. Le \gls{cas} planifié est assuré avec les appareils spécifiquement taskés pour cette \gls{req} ou qui sont ``On-call''. Le \gls{cas} immédiat est assuré par les appareils ``On-call'' ou en re-taskant un appareil déjà prévu dans l'\gls{ato}.

Cfr. \cruderef{ann_casrequest} pour le format de la \gls{req}.

\subsection{Le CAS planifié}

Le besoin de \gls{cas} connu suffisamment longtemps à l'avance sera inclus dans l'\gls{ato} et planifié par toutes les parties. Ces \glspl{req} doivent être envoyées à temps pour permettre une planification efficace.

\begin{e1}
	\itemt{Précédence}{Priorité de la \gls{req}. La précédence sera ajustée au fur et à mesure qu'elle grimpe les échelons de la chaîne de commandement.}
	\itemt{Quantité de détails}{Le quantité de détails inclus dans la \gls{req} est très importante. \textbf{Si possible, l'unité demandeuse devra inclure la description de la cible, sa position, le \gls{tot}, et le reste des données de mission}.}
	\itemt{Timing}{Parfois, tous les détails ne sont pas disponibles avant la parution de l'\gls{ato}. Malgré cela, les \glspl{req} peuvent déjà établir le besoin d'un \gls{cas} pendant une certaine période, l'heure exacte étant précisée plus tard au fur et à mesure que la situation évolue. Une \gls{req} mise à jour après sa publication devra porter le numéro de la \gls{req} initiale.}
	\itemt{Soumission}{Les planificateur à chaque échelon rassemblent les \glspl{req} qui leur sont soumises et les soumettent eux-mêmes à l'échelon supérieur. Les \glspl{req} rejetées sont renvoyées à l'expéditeur avec la raison du rejet.}
	\itemt{Coordination}{Les \glspl{req} approuvées et ayant reçu une priorité sont envoyées au \gls{jaoc} pour inclusion dans l'\gls{ato}.}
\end{e1}

\subsection{Requête immédiate}

Les \gls{req} immédiates sont le résultat d'une situation qui n'a pas été prévue dans le cycle de planification des opérations aériennes. Il est important que les tasking prévus pour réponde à ces \glspl{req} immédiates sont déjà prévus dans l'\gls{ato}.

Puisque les besoins spécifiques ne sont pas connus à l'avance, il se peut que l'armement ou les senseurs embarqués ne correspondent pas de manière optimale aux nécessités de la mission.

\begin{e1}
	\itemt{Dissémination}{La transmission des \glspl{req} immédiates se fait selon le schéma sur la \smallref{immediaterequest} ci-dessous.}
	\itemt{Précédence}{La \gls{req} reçoit une priorité parmi:}
	\begin{e2}
		\item \#1 - Urgence
		\item \#2 - Priorité
		\item \#3 - Routine
	\end{e2}
	\itemt{Update de la situation}{
	Lorsqu'il envoie une \gls{req}, le \ja{} fournit également une update de la situation.}
\todo[inline]{For more information on the situation update, see Chapter V “Execution,” paragraph 2b (3), “Situation Update.”}
	\itemt{Format de la \gls{req}}{
	Les données de mission incluront au minimum les lignes 20 à 28. Cfr. \cruderef{ann_datacard} pour le format de la \gls{req}.}
\end{e1}

\efig{immediaterequest}{Transmission d'une requête CAS immédiate.}


