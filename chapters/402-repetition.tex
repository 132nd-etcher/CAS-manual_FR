\section{Répétitions}

\note{Dans cette section, le mot ``répéter'' est à prendre au même sens que pour le théâtre, où les acteurs ``répètent'' une scène, et non pas au sens ``répéter'' comme ``dire une seconde fois''}

\begin{e1}
	\item Les répétitions sont l'un des aspects le plus souvent ``oublié'' lors de la préparation d'une mission. Ils offrent à tous les participants l'opportunité de se faire une idée du champ de bataille, s'assurer de la totale compréhension du plan par tous, diminue les temps de réaction de chacun, permet d'identifier les zones d'ombre,d e friction ou de conflit parmi les participants. Le fait de ``voir'' les choses permet de les imprimer en mémoire à plus long terme. Les possibilités de la répétition sont limitées par l'imagination, la situation tactique, le temps et les ressources disponibles.
	\item Les \glspl{sop} devrait identifier les types et techniques de répétitions appropriées, ainsi que les standards pour leur exécution. Cette section se concentre sur les points clefs que les participants au \gls{cas} devraient couvrir, ce dont ils devront discuter, et ce qui devra être compris de tous une fois la répétition terminée.
	\item Répétition Joint
	\begin{e2}
		\item{Intentions du \gls{gc}}{Les intentions du \gls{gc} doivent inclure ses intentions quant au \gls{cas}. Le \gls{gc} est conseillé par son \gls{alo} à propos des menaces, de la disponibilité des appareils des combinaisons d'armement potentielles, pour établir un objectif viable et réalisable. Les \glspl{req} doivent clairement décrire l'effet désiré pour accomplir les intentions du \gls{gc}. Souvent, il n'y a pas d'intention propre au \gls{cas}, simplement, le \gls{cas} s'inscrit dans les intentions de tir générale du \gls{gc}.}
		\itemt{Priorités de tir \gls{cas}}{Les \glspl{pof} pour chaque phase de la mission doivent être établies. Pour une sortie \gls{cas}, une prévision de ``qui'' recevra le \gls{cas}, ``quand'' le \gls{cas} est attendu, ``quel'' est l'effet désiré par le \gls{gc} et ``où'' se trouve les observateurs primaires et secondaires sera nécessaire. De plus, à la fin de la répétition, les participants devront avoir une bonne compréhension de:}
		\begin{e3}
			\item Vérification de la grille de coordonnées ou des position des cibles de grande importance, des observateurs, des unités alliées, et des mouvements prévus (par phase)
			\item Critères déclencheurs pour les cibles et leur engagement
			\item \glspl{fscm} / \glspl{acm} leurs implications pour la manoeuvre
			\item Vérifier le plan \gls{sead}
			\item Connectivité des communications
			\item Vérifier la marque des cibles, et, si d'application, la marque des alliés
			\item Quel \ja{} fournira le \gls{tac} des appareils de \gls{cas}
			\begin{e4}
				\item Disponibilité du \gls{faca}
				\item Plan pour utiliser les sorties \gls{cas} en excès (Killboxes ou dégagement vers un autre secteur ou un autre \gls{faca})
				\item Procédures de collection des \glspl{bda} et des \glspl{misrep}
			\end{e4}
			\item Après la répétition, les participants doivent pouvoir communiquer le plan aux unités subordonnées avant le début de l'opération. Durant la répétition, l'\gls{alo} abordera les points suivants:
			\begin{e4}
				\item Confirmation des intentions du \gls{gc}
				\item Nombre de sorties \gls{cas} prévues
				\item Type d'appareil
				\item Armement embarqué
				\item \gls{cas} \gls{tos}
				\item \glspl{cp} et \glspl{ip}
				\item \glspl{acm} / \glspl{fscm}
				\item Plan \gls{sead}
				\item Marques / plan laser
				\item Procédures de marquages de alliés
				\item \glspl{req} approuvées et \glspl{req} refusées
				\item Types de \gls{tac}
			\end{e4}
		\end{e3}
	\end{e2}
	\item Répétition pour l'appui-feu
	\begin{e2}
		\item Les répétitions de l'appui-feu se concentrent sur la synchronisation des différents appuis-feu. Les points suivants doivent être abordés:
		\begin{e3}
			\item Répéter l'exécution du \gls{cas} avec la manoeuvre des éléments au sol et le \gls{jtac}
			\item Identifier les \glspl{fscm} et confirmer qu'elles apportent un plus au plan de manoeuvre
			\item Vérifier la liste collective des cibles pour s'assurer qu'elle inclut les cibles \gls{cas}
			\item Vérifier les positions des coordonnées pour les cible de grande importance
			\item Vérifier que chaque cible \gls{cas} s'est vue assignée un tasking, un but, une méthode et un effet, et que les priorités \gls{cas} sont clairement définies
			\item Vérifier les critères déclencheurs pour chaque cible et pour chaque engagement
			\item Revoir les \glspl{roe} / \glspl{pid}
			\item Répéter les actions \gls{cas} à effectuer lorsque les critères déclencheurs sont remplis
			\item Répéter les points d'observations primaires et secondaires
			\begin{e4}
				\item Identifier les observateurs primaires et secondaires (\gls{jtac}, \gls{faca}, scout, sniper, etc.)
				\item Identifier les unités de force protection
				\item Identifier les routes d'infiltration / exfiltration
				\item Identifier les critères déclencheurs
				\item Identifier les critères qui initient un déplacement
				\item Revoir les considérations météo
				\item Revoir les procédures de nuit
				\item Confirmer le plan d'observation				
			\end{e4}
			\item identifier les unités alliées les plus proches
			\item Vérifier les procédures de marque des alliés
			\item Voir les tactiques \gls{cas} qui seront probablement utilisée (haute, moyenne, basse, très basse altitude)
			\item Répéter les procédures d'engagement pour les cibles \gls{cas}
			\item Revoir le plan de communication
			\begin{e4}
				\item Confirmer les call-signs
				\item Vérifier les \glspl{jtac} primaires / secondaires
				\item Vérifier les mots-code
				\item Effectuer les radio-check des différents réseaux
				\item Vérifier CRYPTO et procédures \gls{comsec}
				\item Revoir les procédures d'authentification
			\end{e4}
			\item Vérifier le guidage terminal pour chaque cible (unité(s) qui tire(nt), nombre de tirs, quantité et type d'appareils, armement standard)
			\item Vérifier et ``déconflictionner'' le plan de mouvement spécifiant quand et où les unités d'appui-feu vont bouger:
			\begin{e4}
				\item \gls{aof} principal
				\item Zones de prises de position
			\end{e4}
			\item Vérifier la méthode d'engagement (``Sur mon ordre'', \gls{tot}, ou ``Quand prêt'')
			\item \glspl{fscm} / \glspl{acm}
			\begin{e4}
				\item Planning ou ordre pour déplacer les \glspl{fscm}
				\item Formel
				\item Informel
			\end{e4}
			\item Identifier les \glspl{cp} / \glspl{ip} et le flux général des appareils de \gls{cas}
			\item Répéter le plan \gls{sead}
			\item Répéter les procédures de marquage des cibles \gls{cas}
			\begin{e4}
				\item Revoir l'intégration des plan de tirs surface et air
				\item Vérifier les unités \gls{idf} disponibles
				\item Revoir les marques, plan \gls{sead} et méthodes de contrôle
				\item Vérifier les positions des unités \gls{idf}
				\item Vérifier les lignes de tir pour les cibles prévues
			\end{e4}
			\item \gls{tot} / \gls{ttt}
			\item Revoir le type de contrôle pour les cibles \gls{cas}
			\begin{e4}
				\item Type 1, 2 ou 3
				\item Vérifier la connectivité entre les observateurs et les contrôleurs
				\item Revoir la procédure d'autorisation de tir pour les cibles \gls{cas}
			\end{e4}
			\item Faire les ajustement nécessaires
			\item \glspl{fscm} comprises par tous
			\item Discuter des positions \gls{idf}
		\end{e3}
	\end{e2}
\end{e1}