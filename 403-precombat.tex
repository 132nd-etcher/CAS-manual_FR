\section{Préparations avant le combat}

\begin{e1}
	\item Les préparations avant le combat permettent au personnel de se préparer pour une mission et donnent au leader une opportunité de s'assurer que le personnel et l'équipement est opérationnel.
	\item La checklist suivante propose un guide pour effectuer cette préparation:
	\begin{e2}
		\itemt{Connaissances essentielles à la mission}{S'assurer que le personnel de chaque élément subordonné comprend la mission, l'objectif et les grandes lignes de la manoeuvre}
		\itemt{Equipement essentiel à la mission}{S'assurer que tout l'équipement est présent et en ordre de fonctionnement. le \gls{jtac} doit prévoir un équipement de communication et un équipement de marquage de rechange.}
		\itemt{Coordination essentielle pour la mission}{S'assurer de la distribution des images ou graphiques dépeignant:}
		\begin{e3}
			\item Grandes lignes de la manoeuvre
			\item \glspl{fscm}
			\item \glspl{acm}
			\item \glspl{nai} et/ou \glspl{tai}
			\item Points de décision et déclencheurs
			\item \glspl{cp} et \glspl{ip} pour les \glspl{fw}
			\item \glspl{ha} et \glspl{bp} pour les \glspl{rw}
			\item Obstacles à la mobilité ou au plan
			\item Procédures de marque de alliés
			\begin{e4}
				\item De jour
				\item De nuit
			\end{e4}
			\item Liste des cibles, graphiques montrant la (zone) cible, chronologie des tirs
			\begin{e4}
				\item \gls{pof} / priorités du \gls{cas}
				\item Cibles prioritaires
				\item Cibles \gls{sead}
			\end{e4}
		\end{e3}
	\end{e2}
\end{e1}