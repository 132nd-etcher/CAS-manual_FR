\thispagestyle{default}

\phantomsection

\addcontentsline{toc}{section}{\protect\numberline{}Préface}

\section*{Préface}

\subsection*{Cadre}

Ce document décrit les procédures \acrfull{cas} en vigueur au sein du \rgt{}.

\subsection*{Trivia}

Ce document est, essentiellement, un résumé et une traduction partielle du \jp{}.

De par sa nature inter-armes, le Close Air Support (CAS) nécessite un ensemble de procédures reconnues et utilisées par toutes les parties; c'est pourquoi les procédures Joint décrites dans le \jp{} sont utilisées.

Ce document implémente les procédures du \jp{} de manière libérale; certaines parties ou spécificités sont omises, soit parce qu'elles ne s'appliquent pas à notre niveau, soit parce que la valeur tactique ou immersive ajoutée ne contrebalancent pas suffisamment la complexité intrinsèque à leur mise en place.

Il est entendu que les procédures implémentées sont celles qui s'appliquent tout particulièrement aux appareils à voilure tournante, et que les procédures propres aux appareils à voilure fixe sont de facto omises ou survolées brièvement.

\textbf{Aucune modification n'a été apportée aux procédures décrites dans le \jp{}, et toute interprétation erronée serait involontaire.}

Ce document trouve son origine dans les événements organisés par la \onethreetwo{}. Le niveau de réalisme atteint lors de ces vols simulés est à couper le souffle (certains membres de la \onethreetwo{} sont J-TAC professionnels), ce qui permet de renforcer le sentiment d'immersion, ainsi que l'efficacité des différents éléments impliqués (une fois les procédures passablement maîtrisées). 