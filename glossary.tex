%\newcommand{\newgls}[4]{% 1: abbrev 2: full name 3: ref 4: desc
%	\newglossaryentry{#3g}{%
%		name={#1},%
%		description={#4}%
%	}%
%	\newglossaryentry{#3}{%
%		type=\acronymtype,%
%		name={#1},%
%		description={#4},%
%		first={#2 (#1)\glsadd{#3g}},%
%		see=[Explication:]{#3g}%
%	}%
%}

\newcommand{\newgls}[4]{% 1: abbrev 2: full name 3: ref 4: desc
	\longnewglossaryentry{#3g}{name={#1}}%
	{\begin{minipage}{\linewidth} #4 \end{minipage}\vskip1ex}%}%
	%{#4}%
	\newglossaryentry{#3}{%
		type=\acronymtype,%
		name={#1},%
		description={#2},%
		first={#2 (#1)\glsadd{#3g}},%
		see=[Explication:]{#3g}%
	}%
}

%\newcommand{\glb}[1]{
%	\begin{minipage}{\linewidth}
%		#1
%	\end{minipage}
%	\glvs
%	}
%\newcommand{\glbe}[1]{
%	\begin{minipage}{\linewidth}
%		\gle
%		#1
%	\gled
%	\end{minipage}
%	\glvs
%}

\newgls{FAC}{Forward Air Controller}{fac}{%
Officier qualifié au contrôle tactique d'un autre appareil pour le \gls{cas} depuis une positions avancée.
}

\newgls{ABFAC}{Airborne Forward Air Controller}{abfac}{%
Plateforme de \gls{fac} à voilure tournante.
}

\newgls{FAC(A)}{Forward Air Controller}{faca}{%
Plateforme de \gls{fac} à voilure fixe.
}

\newgls{TAC(A)}{Tactical Air Coordinator}{taca}{%
Un officier qui coordonne, depuis un appareil en vol, les actions d'un autre appareil engagés dans le support de forces alliées au sol ou navales.
}

\newgls{TACP}{Tactical Air Control Party}{tacp}{%
Un composant opérationnel subordonné de contrôle aérien, qui sert de liaison avec les forces et sol, et qui sert à contrôler des unités aériennes.
}

\newgls{NFA}{No-Fire Area}{nfa}{%
Une zone dans laquelle le tir est interdit.
}

\newgls{RFA}{Restrictive Fire Area}{rfa}{%
Une zone dans laquelle certaines restrictions existent quant à l'emploi d'armement.
}

\newgls{RFL}{Restritive Fire Line}{rfl}{%
Une ligne établie entre deux formations alliées convergentes le long de laquelle le tir est interdit.
}

\newgls{keyhole}{Keyhole}{kh}{%
Le Keyhole est un moyen efficace pour établir un \gls{ip} en l'absence de points de contrôle pré-définis, ou lorsque les points de contrôles déjà établis ne sont plus adaptés à la situation.\\[1ex]%
Lorsqu'un appareil \gls{cas} se rapporte au \ja{} sur le \gls{cp} (=check-in), le \ja{} définit un point ``Echo''.\\[1ex]%
Ce point Echo se trouve généralement sur la cible (ligne 6 de la \gls{9l}).\\[1ex]%
Le \ja{} utilise ensuite ce point Echo pour définir des positions d'attente en donnant une direction et une distance à partir du point Echo.\\[1ex]%
Cette distance est à considérer comme une distance \textbf{minimale} à partir du point Echo.
}

\newgls{offset}{Décalage}{of}{%
L'offset indique au pilote la direction (gauche ou droite) vers laquelle il peut manoeuvrer lors de son attaque (de l'\gls{ip} vers la cible).\\[1ex]%
Le \ja{} utilise un offset pour faciliter la coordination avec les autres tirs dans la zone, pour aligner l'appareil pour l'attaque ou pour l'egress, ou pour éloigner l'appareil de menaces potentielles.\\[1ex]%
L'offset permet de réguler l'angle d'attaque sans imposer un \gls{fah}.
}

\newgls{KB}{Killbox}{kb}{%
Une Killbox est une zone tridimensionnelle utilisée pour faciliter l'intégration de l'appui-feu dans l'espace aérien. Une Killbox est une \gls{fscm} associée à une \gls{acm}. Il s'agit d'une \gls{fscm} permissive, créée par le \gls{gc} approprié, qui peut éventuellement contenir d'autres mesures de contrôle restrictives (\gls{acm} ou \gls{fscm}) dans la zone couverte. Pour les procédures Killbox, voyez les \glspl{sop} de l'opération.\\[1ex]%
Une Killbox est une zone tridimensionnelle utilisée pour faciliter l'intégration de l'appui-feu dans l'espace aérien. Une Killbox est une \gls{fscm} associée à une \gls{acm}. Il s'agit d'une \gls{fscm} permissive, créée par le \gls{gc} approprié, qui peut éventuellement contenir d'autres mesures de contrôle restrictives (\gls{acm} ou \gls{fscm}) dans la zone couverte. Pour les procédures Killbox, voyez les \glspl{sop} de l'opération.\\[1ex]%
Lorsqu'elle est créée, la Killbox a pour objectif premier de permettre l'emploi d'armement létal contre des cibles au sol sans devoir passer par le \gls{gc} local et sans \gls{tac}.\\[1ex]%
L'établissement d'une Killbox n'implique pas automatiquement l'autorisation de tir. Tous les appareils qui effectuent une mission \gls{ai} dans une Killbox le feront selon les règles établies pour la \gls{pid} et les \glspl{cde} et en accord avec les \glspl{roe} d'application ainsi que les \gls{spins}.
}

\newgls{CDE}{Collateral Damage Estimation}{cde}{%
Estimation des dégâts collatéraux possibles.
}

\newgls{FFA}{Free Fire Area}{ffa}{%
Une zone spécifique dans laquelle le tir est autorisé sans coordination préalable.
}

\newgls{FSCL}{Fire Support Coordination Line}{fscl}{%
Mesure de coordination du tir mise en place par le \gls{gc} au sein d'une zone d'opération.\\[1ex]%
La FSCL est une ligne se trouvant après la \gls{flot}, qui délimite la zone dans laquelle le \gls{gc} contrôle les tirs.\\[1ex]%
L'emploi d'armement n'est pas d'office autorisé après une FSCL; c'est seulement que cela ne dépend plus du \gls{gc} qui a établi la FSCL.
}

\newgls{CFL}{Coordinated Firing Line}{cfl}{%
Ligne au delà de laquelle le tir est autorisé sans coordination avec l'échelon supérieur.
}

\newgls{NAI}{Named Area of Interest}{nai}{%
Zone particulière à laquelle on aura attribué un nom (ou qui porte déjà un nom, par ex. une ville).
}

\newgls{TAI}{Target Area of Interest}{tai}{%
Zone cible particulière.
}

\newgls{AOF}{Azimuth Of Fire}{aof}{%
Direction dans laquelle tire une unité.
}

\newgls{IFF}{Identitification Friend or Foe}{iff}{%
Système électronique embarqué permettant d'interroger à distance un appareil et de l'identifier comme ami.
}

\newgls{PID}{Positive Identification}{pid}{%
Identification visuelle, électronique, \gls{iff} ou autre obtenue par l'observation et l'analyse des caractéristiques de la cible.
}

\newgls{CP}{Contact Point}{cp}{%
Point sur lequel, lorsqu'elle arrive dessus, une unité est censée en contacter une autre.
}

\newgls{POF}{Priorities Of Fire}{pof}{%
Priorités de tir.
}

\newgls{COE}{Concept Of Employment}{coe}{%
Façon d'employer quelque chose.
}

\newgls{JIPOE}{Joint Intelligence Preparation of the Operational Environment}{jipoe}{%
Le processus analytique par lequel les organisations de renseignement produisent des estimations et autres pour aider au processus de prise de décision pas le \gls{jfc}.
}

\newgls{CAS request}{CAS request}{req}{%
Une requête \gls{cas} est une demande de \gls{cas} émise par une unité alliée au sol pour obtenir un support aérien. Ce support peut être planifié à l'avance (``Pre-planned \gls{cas}''), ou immédiat (``On-call \gls{cas}'').
}

\newgls{OPLAN}{Operational Plan}{oplan}{%
1. Tout plan pour la conduite d'une opération militaire préparée en réponse à une situation effective ou prévue.\\[1ex]%
2. Un plan Joint complet et détaillé contenant la description complète du \gls{conops}, toutes les annexes d'application au plan, et une chronologie prévue de l'emploi de la force et du déploiement.
}

\newgls{TOS}{Time On Station}{tos}{%
Heure d'arrivée prévue sur la zone cible, de patrouille, sur l'objectif, etc.
}

\newgls{ACM}{Airspace Control Measure}{acm}{%
Directive visant à rendre l'utilisation des unités aériennes efficaces et aptes à remplir leur mission tout permettant le transit sécurisé dans l'espace aérien.
}

\newgls{ACA}{Airspace Coordination Area}{aca}{%
Un bloc tridimensionnel d'espace aérien se trouvant dans une zone cible, défini par l'autorité appropriée, dans lequel les appareils alliés sont normalement hors de danger d'être touché par un tir ami provenant du sol (artillerie, principalement, ou Kakane, parfois).
}


\newgls{AI}{Air Interdiction}{ai}{%
L'interdiction aérienne consiste à utiliser une unité aérienne pour attaquer des cibles ennemies tactiques au sol qui ne trouvent pas à proximité de forces au sol alliées.\\[1ex]%
L'interdiction diffère du \gls{cas} en cela qu'elle ne soutient pas directement les opérations des forces au sol, et qu'elle n'est pas coordonnée aussi étroitement avec ces dernières.\\[1ex]%
A la différence du bombardement stratégique, l'interdiction n'est pas une campagne aérienne à part entière, de par son objectif ultime qui reste le soutien des forces au sol alliées, là où le bombardement stratégique vise à défaire l'ennemi par la puissance aérienne uniquement.\\[1ex]%
L'objectif de l'interdiction est de retarder, perturber ou détruire les forces ennemies ou la logistique en route vers la zone de combat avant qu'elles en deviennent un facteur pour les forces alliées. Même à ce degré, une distinction est faite entre l'interdiction stratégique et l'interdiction tactique; l'interdiction stratégique est prévue à grande échelle et sur le long terme, là où l'interdiction tactique agit de manière rapide et localisée.\\[1ex]%
Cfr. \cruderef{agops} pour une illustration des différents types d'opérations air-sol.
}

\newgls{DAS}{Deep Air Support}{das}{%
Le DAS englobe le \gls{ai}, le \gls{ar} et le \gls{scar}, trois types de missions qui, à l'inverse du \gls{cas}, sont dirigées contre les unités ennemies au sol qui ne trouvent pas à proximité des forces au sol alliées.
}

\newgls{SCAR}{Air Interdiction}{scar}{%
Le SCAR est une mission dont l'objectif est l'acquisition et le rapportage de cible pour le \gls{das}. Les missions SCAR coordonnent et peuvent marquer les cibles pour les missions \gls{ai}, ou repérer précisément les cibles pour les missions \gls{ar}. Le SCAR ne doit pas être confondu avec le \gls{faca}.\\[1ex]%
\begin{itemize}
	\item Le SCAR ne nécessite pas une qualification \gls{faca} pour le \gls{tac} des missions \gls{das}.
	\item Le SCAR fournit des cibles, des positions, des descriptions, des informations quant à la météo ou la menace.
	\item Le SCAR peut fournir la marque mais pas l'autorisation de tir.
	\item Le SCAR confirme ou découvre des menaces air-sol.
	\item Le SCAR assiste à l'établissement de la \gls{bda}.
	\item Le SCAR diffère d'une mission de reconnaissance de par le fait qu'il découvre les cibles et coordonne leurs destruction, et qu'il sera typiquement armé et équipé de manière à améliorer la désignation des cibles.\\[1ex]%
\end{itemize}
Le SCAR est particulièrement utile dans un environnement riche en cibles. Un appareil en mission SCAR qui se trouve à court de munitions peut continuer sa mission de recherche et de coordination.\\[1ex]%
Cfr. \cruderef{agops} pour une illustration des différents types d'opérations air-sol.
}

\newgls{AR}{Armed Reconnaissance}{ar}{%
La reconnaissance armée est utilisée quand la position exacte de la cible est inconnue, et que le pilote devra localiser et engager les cibles potentielles pour accomplir l'objectif.\\[1ex]%
Ces missions impliquent que l'appareil sera exposé pendant de longues période en territoire ennemi, à la recherche de cibles potentielles.\\[1ex]%
\begin{itemize}
	\item Identifier des cibles dont la position est inconnue et les engager avant qu'elles ne deviennent un facteur.
	\item Empêcher l'ennemi d'effectuer un mouvement non détecté ou d'accéder à une zone clef.
	\item Fournir un avertissement précoce quant à la position et aux intentions de l'ennemi.
	\item Empêcher ou dégrader la mobilité de l'ennemi.
	\item Reconnaître une grande étendue de terrain qui serait moins facilement explorée par les forces au sol amies.
	\item Attaquer des cibles dont la destruction doit se faire rapidement (menaces en mouvement vers le front, par exemple).
	\item Soutien lors des opérations de sécurité, soit comme unité de soutien du support ou comme couverture, garde ou écran principal.
\end{itemize}
Cfr. \cruderef{agops} pour une illustration des différents types d'opérations air-sol.
}

\newgls{BAI}{Battlefield Air Interdiction}{bai}{%
Cfr \gls{ai}
}

\newgls{COMSEC}{Communications Security}{comsec}{%
Tout ce qui a trait à la sécurisation des communications alliées.
}

\newgls{FSCM}{Fire Support Coordination Measure}{fscm}{%
Directive mise en place pour coordonner le tir de support (artillerie).
}

\newgls{FSCOORD}{Fire Support Coordinator}{fscoord}{%
Officier en charge de la coordination du tir de support (artillerie).
}

\newgls{COA}{Course Of Action}{coa}{%
Ligne de conduite, façon de faire.
}

\newgls{5-Line}{5-Line}{5l}{%
La 5-Line est une version résumée de la \gls{9l}, réservée aux hélicoptères, et centrées sur les forces alliées.
}

\newgls{9-Line}{9-Line}{9l}{%
Format standard du briefing donné par le J-TAC/FAC(A).
}

\newgls{FLOT}{Forward Line Of Troops}{flot}{%
Ligne de front, là où les troupes alliées sont les plus avancées.
}

\newgls{FRAGO}{Fragmentary Order}{frago}{%
Forme abrégée de l'\gls{opord} publiée journellement.
}

\newgls{BOC}{Bombs On Coordinates}{boc}{%
Méthode d'attaque consistant à attaquer la cible au moyen d'une munition intelligente sur des coordonnées, parfois sans jamais acquérir la cible. La méthode d'attaque BOC implique de rentrer \textbf{directement} les coordonnées de la cible dans le système lors du \gls{cas} brief, et d'effectuer le read-back de la ligne 6 en \textbf{relisant les coordonnées entrées dans le système.}
}

\newgls{HMCS}{Helmet Mounted Cueing System}{hmcs}{%
Système de visée monté sur le casque (HMS).
}

\newgls{FLIR}{Forward Looking Infrared}{flir}{%
Équipement embarqué permettant de voir en infrarouge devant l'appareil.
}

\newgls{FOV}{Field Of View}{fov}{%
Champ de vision.
}

\newgls{IDF}{Indirect Fire}{idf}{%
Feu indirect, c-à-d artillerie.
}

\newgls{TRP}{Target Reference Point}{trp}{%
Point de référence cible (NAV TGT dans le PVI800).
}

\newgls{BOT}{Bombs On Target}{bot}{%
Méthode d'attaque consistant à attaquer après l'avoir acquise visuellement ou via les senseurs embarqués.
}

\newgls{stack}{stack}{stack}{%
Mise en attente des appareils de manière sécurisée.
}

\newgls{GPS}{Global Positionning System}{gps}{%
Système permettant d'obtenir les coordonnées d'un point sur la terre au moyen d'une constellation de satellites.
}

\newgls{LRF}{Laser Range Finder}{lrf}{%
Laser servant à calculer une distance.
}

\newgls{JDAM}{Joint Direct Attack Munition}{jdam}{%
Un JDAM est un kit qui convertit une bombe lisse en bombe guidée \gls{gps}.
}

\newgls{MSL}{Mean Sea Level}{msl}{%
Niveau moyen de la mer.
}

\newgls{HVT}{High Value Target}{hvt}{%
Cible de grande importance.
}

\newgls{CSAR}{Combat Search and Rescue}{csar}{%
Recherche et sauvetage de combat.
}

\newgls{OPCON}{Operationnal Control}{opcon}{%
Entité qui exerce le contrôle opérationnel (stratégique), par opposition au contrôle tactique (\gls{tacon}).
}

\newgls{SEAD}{Suppression of Ennemy Air Defence}{sead}{%
Aussi appelé ``Wild Weasel'', ou ``Iron Hand''. Action visant à neutraliser les défenses anti-aériennes ennemies, pas uniquement les \glspl{sam} ou \glspl{aaa}, mais également les \glspl{ewr} et le \gls{ccc}. Cette action peut être effectuée au moyen d'armement, ou de moyens de guerre électronique.
}

\newgls{CID}{Combat Identification}{cid}{%
Capacité de différencier les cibles potentielles, mobiles ou fixes, dans de grandes zones, comme amies, ennemies ou neutre, dans un temps très court, avec un grand degré de certitude, et à la distance suffisante, pour permettre les décisions d'emploi de la force.
}

\newgls{AOD}{Air Operation Directive}{aod}{%
Directive émise par le \gls{jfacc} concernant la conduite des opération aériennes.%
}

\newgls{ROE}{Rule of Engagement}{roe}{%
Décrit les circonstances, les conditions, la manière et la proportion dans lesquelles l'usage de la force est autorisé.%
}

\newgls{TGO}{Terminal Guidance Operation}{tgo}{%
Action de fournir aux appareils ou aux armes des informations supplémentaires relatives à la cible par la voix, l'électronique, les communications visuelles ou des moyens mécaniques.%
}

\newgls{TAC}{Terminal Attack Control}{tac}{%
Autorité qui contrôle la manoeuvre et le tir.%
}

\newgls{BDA}{Battle Damage Assessment}{bda}{%
Évaluation des dégâts infligés à la cible, ou aux cibles, suite à un engagement.%
}

\newgls{J-TAC}{Joint Terminal Attack Controller}{jtac}{%
Un membre du personnel qualifié qui contrôle, depuis une position au sol avancée, les appareils engagés en opération de CAS ou d'autres types d'opération offensive.%
}

\newgls{GCAS}{Ground alert Close Air Support}{gcas}{\gls{cas} à la demande, avec les appareils en attente sur le parking.}

\newgls{XCAS}{Air alert Close Air Support}{xcas}{\gls{cas} à la demande, avec les appareils en attente en vol.}

\newgls{GC}{Ground Commander}{gc}{Commandant des troupes au sol.}%

\newgls{JFC}{Joint Force Commander}{jfc}{Commandant de la coalition alliée.}%

\newgls{JFACC}{Joint Force Air Component Commander}{jfacc}{Commandant de la composante aérienne de la coalition alliée.}%

\newgls{AOB}{Air Order of battle}{aob}{%
Escadrons, types d'appareils et base d'attache des appareils de la coalition alliée.%
}

\newgls{ATO}{Air Tasking Order}{ato}{%
Document reprenant toutes les sorties aériennes prévues dans les 24 prochaines heures.%
}

\newgls{ACO}{Airspace Control Order}{aco}{%
Document reprenant toutes les mesures de contrôle et de coordination de l'espace aérien en vigueur.%
}

\newgls{SPINS}{Special Instructions}{spins}{%
Document reprenant toutes les procédures spécifiques à une opération, un escadron, une sortie, etc.%
}

\newgls{OPORD}{Operations Order}{opord}{%
Plan visant à assister les unités subordonnées lors de l'exécution de leur tâche. Ce plan contient la situation à laquelle les unités doivent faire face, la mission de l'unité, ainsi que ce que l'on attend de l'unité. Cfr \url{https://en.wikipedia.org/wiki/Operations_order} pour un exemple d'OPORD.%
}

\newgls{SOP}{Standard Operating Procedure}{sop}{%
Ensemble d'instructions visant à uniformiser les procédures en vigueur.%
}

\newgls{CONOPS}{Concept of Operation}{conops}{%
Document qui décrit de manière claire et concise les objectifs du \gls{jfc} et ses intentions quant à la façon de les accomplir.%
}

\newgls{COMPLAN}{Communications Plan}{complan}{%
Plan de communications (réseaux, canaux, fréquence, procédures).%
}

\newgls{SA}{Situationnal Awareness}{sa}{%
La perception de la situation est la capacité de se représenter l'environnement autour de soi, et à s'en fabriquer une image mentale persistante. Un pilote chevronné parviendra également à projeter cet environnement dans le temps, et à réagir au danger avant qu'il ne devienne un facteur.%
}

\newgls{LGB}{Laser Guided Bomb}{lgb}{%
Bombe guidée par laser.
}

\newgls{LGW}{Laser Guided Weapon}{lgw}{%
Arme guidée par laser.
}

\newgls{LTD}{Laser Target Designator}{ltd}{%
Désignateur de cible laser.
}

\newgls{LSS}{Laser Spot Search}{lss}{%
Recherche de point laser.
}

\newgls{FAH}{Final Attack Heading}{fah}{%
Cap que doit suivre l'appareil qui attaque pour se diriger vers la cible et engager.
Le \ja{} peut imposer une cap d'attaque pour plusieurs raisons: augmenter la sécurité des troupes alliées, permettre au \ja{} de repérer plus facilement l'appareil qui attaque, aider le pilote à repérer la cible, minimiser les dommages collatéraux, fournir un cône de sécurité laser, ou faciliter la coordination avec les autres tirs.\\[1ex]%
Le cap d'attaque final peut être donnée sous forme de fourchette, pour donner plus de flexibilité au pilote.
}

\newgls{LST}{Laser Spot Tracker}{lst}{%
Détecteur de point laser.
}

\newgls{PGM}{Precision Guided Munition}{pgm}{%
Munition intelligente guidée.
}

\newgls{NVG}{Night Vision Goggles}{nvg}{%
Lunettes de vision nocturne (\gls{ir}).
}

\newgls{ISR}{Intelligence, Surveillance, Reconnaissance}{isr}{%
Espionnage, surveillance et reconnaissance.
}

\newgls{ELINT}{Electronic Intelligence}{elint}{%
Espionnage électronique.
}

\newgls{IP}{Initial Point}{ip}{%
Point de départ de l'attaque avant de se diriger vers la cible.
}

\newgls{EW}{Electronic Warfare}{ew}{%
Guerre électronique.
}

\newgls{EO}{Electro-Optics}{eo}{%
Électro-optique (=télévision).
}

\newgls{GRG}{Gridded Reference Graphic}{grg}{%
Graphique de référence quadrillé.
}

\newgls{LOS}{Line Of Sight}{los}{%
Ligne de vue.
}

\newgls{PLA}{Post Launch Annulation}{pla}{%
Annulation de l'attaque après que l'arme ait été tirée.
}

\newgls{TLE}{Target Location Error}{tle}{%
Ligne de vue.
}

\newgls{ALO}{Air Liaison Officier}{alo}{%
L'officier en charge de conseiller le \gls{gc} quant à l'emploi de la composante aérienne dans sa stratégie.
}

\newgls{SNFS}{Naval Fire Support}{nsfs}{%
Tir de support depuis les unités navales.
}

\newgls{JAOC}{Joint Air Operation Center}{jaoc}{%
Centre de contrôle des opérations aériennes de la coalition, l'échelon de commandement le plus haut de la composante air de la coalition.
}

\newgls{JOC}{Joint Operation Center}{joc}{%
	Centre de contrôle des opérations de la coalition, l'échelon de commandement le plus haut de la coalition.
}

\newacronym{misrep}{MISREP}{Mission Report}
\newacronym{cc}{C2}{Command and Control}
\newacronym{ccc}{C3}{Command, Control and Communications}
\newacronym{cas}{CAS}{Close Air Support}
\newacronym{awacs}{AWACS}{Airborne Warning And Control System}
\newacronym{rw}{RW}{Rotary-Wing}
\newacronym{fw}{FW}{Fixed-Wing}
\newacronym{farp}{FARP}{Forward Arming and Refueling Point}
\newacronym{tot}{TOT}{Time On Target}
\newacronym{ttt}{TTT}{Time To Target}
\newacronym{bp}{BP}{Battle Position}
\newacronym{ha}{HA}{Holding Area}
\newacronym{fp}{FP}{Firing Point}
\newacronym{tacon}{TACON}{Tactical Control}
\newacronym{noe}{NOE}{Nap Of the Earth}
\newacronym{rpg}{RPG}{Rocket Propelled Grenade}
\newacronym{ttp}{TTP}{Tactics, Techniques, and Procedures}
\newacronym{sam}{SAM}{Surface to Air Missile}
\newacronym{aaa}{AAA}{Anti Air Artillery}
\newacronym{ewr}{EWR}{Early Warning Radar}
\newacronym{ft}{ft}{feet [pieds, 1 pied=30,48cm]}
\newacronym{recce}{RECCE}{Reconnaissance}
\newacronym{sparkle}{sparkle}{Marque infrarouge}
\newacronym{ir}{IR}{Infrarouge}
\newacronym{ks}{Killer Scout}{Sobriquet pour le \gls{faca}}
\newacronym{abort}{ABORT}{Appel: "Cessez action/attaque/mission"}
\newacronym{clearedhot}{CLEARED HOT}{Appel (contrôle de types 1 et 2): "Autorisé à larguer de l'armement pour cette passe"}
\newacronym{continue}{CONTINUE}{Appel: "Poursuivez la manoeuvre en cours" (n'implique pas l'autorisation de tir)}
\newacronym{continuedry}{CONTINUE DRY}{Appel: "Poursuivez la manoeuvre en cours, tir interdit" (pour le contrôle de type 3, le \gls{tac} doit utiliser "Type 3, CONTINUE DRY") (utilisé pour l'entraînement ou le "Show of force")}
\newacronym{clearedeng}{CLEARED TO ENGAGE}{Appel (contrôle de Type 3): "L'appareil ou la formation est autorisée à commencer l'engagement (tirer) selon les paramètres établis avec le \gls{tac}"}
\newacronym{commencing}{COMMENCING ENGAGEMENT}{Appel (contrôle de Type 3) émis par le pilote pour signaler le début d'une attaque de Type 3}
\newacronym{complete}{ENGAGEMENT COMPLETE}{Appel (contrôle de Type 3) émis par le pilote pour signaler la fin d'un engagement de Type 3}
\newacronym{ipinbound}{IP INBOUND}{Appel émis par le pilote pour signaler qu'il se trouve sur l'\gls{ip}}
\newacronym{in}{IN}{Appel émis par le pilote pour signaler qu'il est sur le point d'engager la cible et atteint l'autorisation (\gls{clearedhot})}
\newacronym{captured}{CAPTURED}{Appel émis par le pilote pour signaler qu'il a identifié la cible est qu'il l'a verrouillée avec un senseur embarqué}
\newacronym{tally}{TALLY}{Appel émis par le pilote pour signaler qu'il a identifié la cible visuellement}
\newacronym{contact}{CONTACT}{Appel émis par le pilote pour signaler qu'il a visuellement identifié un point de référence ou une marque}
\newacronym{asap}{ASAP}{Le plus vite possible (As Soon As Possible)}
\newacronym{nlt}{NLT}{Pas plus tard que (Not Later Than)}


\newgls{Type 1}{Contrôle de Type 1}{typeone}{%
	Cfr. \cref{typeone}: \nameref{typeone}.
}
\newgls{Type 2}{Contrôle de Type 2}{typetwo}{%
	Cfr. \cref{typetwo}: \nameref{typetwo}.
}
\newgls{Type 3}{Contrôle de Type 3}{typethree}{%
	Cfr. \cref{typethree}: \nameref{typethree}.
}

\newgls{UR}{Univers Radio}{ur}{%
	Logiciel de communication intégré à TeamSpeak 3 et à DCS, créé par le commandant du 75th vFS, et permettant de simuler un environnement radiophonique réel. Cfr. http://tacnoworld.fr/universradio/.
}

\newgls{GINGERBREAD}{}{ginger}{%
	Mot code utilisé sur un réseau radio pour signaler aux autres stations sur le réseau qu'une station inconnue utilise votre call-sign (risque de spoofing).
}

\newgls{BEADWINDOW}{}{bead}{%
	Mot code utilisé sur un réseau radio pour signaler à une station radio qu'elle diffuse des informations importantes sur un réseau non-sécurisé (positions des unités amies, plan, intentions, \ldots{} -> brèche de \gls{comsec}). La \emph{seule} réponse admise à un BEADWINDOW est: ``Roger out''.
}

\newgls{TACS}{Theater Air Control System}{tacs}{%
	Réseau servant à coordonner les différentes agences de contrôle de l'espace aérien sur le théâtre d'opération.
}

\newgls{TAD}{Tactical Air Direction}{tad}{%
	Réseau servant à contrôler les appareils.
}

\newgls{ASOC}{Air Support Operations Center}{asoc}{%
	Agence de contrôle principale pour les appareils dont l'objectif est le support direct des forces au sol.
}

\newgls{INFLTREP}{In-Flight Report}{infltrep}{%
	Rapport d'information quant au déroulement d'une mission effectuée par un appareil envoyé durant le vol.
}

\newgls{AOC}{Air Operations Center}{aoc}{%
	Centre de contrôle des opérations aériennes.
}

\newgls{IRC}{Internet Relay Chat}{irc}{%
	Réseau de communications écrites passant par internet.
}

\newgls{game-plan}{}{gp}{%
	Le game-plan décrit la manière dont le \gls{jtac} envisage d'attaquer la cible.
}

\newgls{CAS-brief}{}{cb}{%
	Le CAS-brief est un briefing standardisé permettant au \gls{jtac} de transmettre rapidement les informations importantes au pilote.
}

\newgls{HQ}{Headquarters}{hq}{%
	Quartier général.
}


\newgls{HHQ}{Higher Headquarter}{hhq}{%
	Quartier général de l'échelon hiérarchique supérieur.
}


\newgls{CCIR}{Commander's Critical Information Requirement}{ccir}{%
	Toute information que le commandant considère comme cruciale pour son processus de prise de décision.
}

\newgls{JFO}{Joint Fire Observer}{jfo}{%
	Un soldat qualifié qui peut demander, ajuster et contrôler le tir sol-sol, fournir des informations de ciblage pour les contrôles de Type 2 et 3, et effectuer de manière autonome des opérations de \gls{tac}.
}

\newgls{IRL}{In Real Life}{irl}{%
	Indique que l'on fait référence à quelque chose ``dans la vraie vie'', par opposition à la simulation (DCS). Inverse: \gls{ig}.
}

\newgls{IG}{In Game}{ig}{%
	Indique que l'on fait référence à quelque chose qui est simulé (DCS). Inverse: \gls{irl}.
}

\newgls{HIDACZ}{High Density Airspace Control Zone}{hidacz}{%
	Zone de l'espace aérien dans laquelle on s'attend à devoir gérer un grand nombre d'appareils ou de tirs.\\[1ex]%
	Une HIDACZ a généralement des limites qui correspondent à des points remarquables sur le terrain ou à des aides à la navigation.\\[1ex]%
	Pour plus d'informations sur les HIDACZ, cfr. \cite{jp352}.
}

\newgls{ROZ}{Restricted Operation Zone}{roz}{%
	Une ROZ est une zone en trois dimensions restreint l'usage d'une partie de l'espace aérien à un type particulier d'opération (CAS, UA, AI, ...).
}

\newgls{ROA}{Restricted Operation Area}{roa}{%
	Cfr. \gls{roz}.
}

\newgls{MRR}{Minimum Risk Route}{mrr}{%
	Un corridor en trois dimensions qui définit une route recommandée pour les \glspl{fw}, et qui est la route présentant le moins de danger pour traverser la zone de combat.
}

\newgls{CA}{Coordination Altitude}{ca}{%
	Une mesure qui utilise une délimitation en altitude pour coordonner les appareils dans l'espace aérien.
}

\newgls{SAAFR}{Standard Use Army Aircraft Flight Route}{saafr}{%
	Une SAAFR est établie en dessous de la \gls{ca} pour faciliter le transit des \glspl{rw}.\\[1ex]%
	Cette route se trouve derrière la \gls{flot}, en territoire allié, et permet aux \glspl{rw} de se déplacer sans avoir à être contrôlés par une agence \gls{cc}.
}

\newgls{ADA}{Air Defence Artillery}{ada}{%
	Artillerie de défense anti-aérienne.\\[1ex]%
	Le terme ``ADA'' fait référence aux AAA alliées.
}

\newgls{MANPAD}{Man Portable Air Defence System}{manpad}{%
	Système de défense anti-aérienne portable.
}

\newgls{AOS}{Aircraft On Station}{aos}{%
	Appareil en station sur un point ou une zone donnée.
}