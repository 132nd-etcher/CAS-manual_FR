\newcommand{\newgls}[4]{% 1: abbrev 2: full name 3: ref 4: desc
	\newglossaryentry{#3g}{%
		name={#1},%
		description={#4}%
	}%
	\newglossaryentry{#3}{%
		type=\acronymtype,%
		name={#1},%
		description={#2},%
		first={#2 (#1)\glsadd{#3g}},%
		see=[Explication:]{#3g}%
	}%
}

\newgls{FRAGO}{Fragmentary Order}{frago}{%
Forme abrégée de l'OPORD publiée journellement
}

\newgls{BOC}{Bombs On Coordinates}{boc}{%
Méthode d'attaque consistant à attaquer la cible au moyen d'une munition intelligente sur des coordonnées, parfois sans jamais acquérir la cible. La méthode d'attaque BOC implique de rentrer \textbf{directement} les coordonnées de la cible dans le système lors du CAS brief, et d'effectuer le read-back de la ligne 6 en \textbf{relisant les coordonnées entrées dans le système}
}

\newgls{HMCS}{Helmet Mounted Cueing System}{hmcs}{%
Système de visée monté sur le casque (HMS)
}

\newgls{FLIR}{Forward Looking Infrared}{flir}{%
Équipement embarqué permettant de voir en infrarouge devant l'appareil
}

\newgls{FOV}{Field Of View}{fov}{%
Champ de vision
}

\newgls{IDF}{Indirect Fire}{idf}{%
Feu indirect, c-à-d artillerie
}

\newgls{TRP}{Target Reference Point}{trp}{%
Point de référence cible (NAV TGT dans le PVI800)
}

\newgls{BOT}{Bombs On Target}{bot}{%
Méthode d'attaque consistant à attaquer après l'avoir acquise visuellement ou via les senseurs embarqués
}

\newgls{stack}{stack}{stack}{%
Mise en attente des appareils de manière sécurisée
}

\newgls{GPS}{Global Positionning System}{gps}{%
Système permettant d'obtenir les coordonnées d'un point sur la terre au moyen d'une constellation de satellites
}

\newgls{LRF}{Laser Range Finder}{lrf}{%
Laser servant à calculer une distance
}

\newgls{JDAM}{Joint Direct Attack Munition}{jdam}{%
Un JDAM est un kit qui convertit une bombe lisse en bombe guidée GPS
}

\newgls{MSL}{Mean Sea Level}{msl}{%
Niveau moyen de la mer
}

\newgls{HVT}{High Value Target}{hvt}{%
Cible de grande importance
}

\newgls{CSAR}{Combat Search and Rescue}{csar}{%
Recherche et sauvetage de combat
}

\newgls{OPCON}{Operationnal Control}{opcon}{%
Entité qui exerce le contrôle opérationnel (stratégique), par opposition au contrôle tactique (TACON)
}

\newgls{SEAD}{Suppression of Ennemy Air Defence}{sead}{%
Aussi appelé ``Wild Weasel'', ou ``Iron Hand''. Action visant à neutraliser les défenses anti-aériennes ennemies, pas uniquement les SAMs ou AAA, mais également mes EWRs et le C3. Cette action peut être effectuée au moyen d'armement, ou de moyens de guerre électronique
}

\newgls{CID}{Combat Identification}{cid}{%
Capacité de différencier les cibles potentielles, mobiles ou fixes, dans de grandes zones, comme amies, ennemies ou neutre, dans un temps très court, avec un grand degré de certitude, et à la distance suffisante, pour permettre les décisions d'emploi de la force
}

\newgls{AOD}{Air Operation Directive}{aod}{%
Directive émise par le JFACC concernant la conduite des opération aériennes%
}

\newgls{AI}{Air Interdiction}{ai}{%
Aussi appelé DAS, pour ``Deep Air Support''. L'interdiction aérienne consiste à frapper des cibles au sol ennemies qui ne sont pas à proximité de troupes alliées. L'AI diffère du bombardement de par son but, qui reste le soutien des troupes au sol alliées%
}

\newgls{ROE}{Rule of Engagement}{roe}{%
Décrit les circonstances, les conditions, la manière et la proportion dans lesquelles l'usage de la force est autorisé%
}

\newgls{TGO}{Terminal Guidance Operation}{tgo}{%
Action de fournir aux appareils ou aux armes des informations supplémentaires relatives à la cible par la voix, l'électronique, les communications visuelles ou des moyens mécaniques%
}

\newgls{TAC}{Terminal Attack Control}{tac}{%
Autorité qui contrôle la manœuvre %
}

\newgls{BDA}{Battle Damage Assessment}{bda}{%
Évaluation des dégâts infligés à la cible, ou aux cibles, suite à un engagement%
}

\newgls{FAC(A)}{Forward Air Controller}{faca}{%
Un pilote spécialement qualifié qui contrôle, depuis les airs, d'autres appareils engagés en opération de CAS%
}

\newgls{J-TAC}{Joint Terminal Attack Controller}{jtac}{%
Un membre du personnel qualifié qui contrôle, depuis une position au sol avancée, les appareils engagés en opération de CAS ou d'autres types d'opération offensive%
}

\newgls{GCAS}{Ground alert Close Air Support}{gcas}%
{CAS à la demande, avec les appareils en attente sur le parking}%

\newgls{XCAS}{Air alert Close Air Support}{xcas}%
{CAS à la demande, avec les appareils en attente en vol}

\newgls{GC}{Ground Commander}{gc}{Commandant des troupes au sol}%

\newgls{JFC}{Joint Force Commander}{jfc}{Commandant de la coalition alliée}%

\newgls{JFACC}{Joint Force Air Component Commander}{jfacc}{Commandant de la composante aérienne de la coalition alliée}%

\newgls{AOB}{Air Order of battle}{aob}{%
Escadrons, types d'appareils et base d'attache des appareils de la coalition alliée%
}

\newgls{ATO}{Air Tasking Order}{ato}{%
Document reprenant toutes les sorties aériennes prévues dans les 24 prochaines heures%
}

\newgls{ACO}{Airspace Control Order}{aco}{%
Document reprenant toutes les mesures de contrôle et de coordination de l'espace aérien en vigueur%
}

\newgls{SPINS}{Special Instructions}{spins}{%
Document reprenant toutes les procédures spécifiques à une opération, un escadron, une sortie, etc%
}

\newgls{OPORD}{Operations Order}{opord}{%
Plan visant à assister les unités subordonnées lors de l'exécution de leur tâche. Ce plan contient la situation à laquelle les unités doivent faire face, la mission de l'unité, ainsi que ce que l'on attend de l'unité. Cfr \url{https://en.wikipedia.org/wiki/Operations_order} pour un exemple d'OPORD%
}

\newgls{SOP}{Standard Operating Procedure}{sop}{%
Ensemble d'instructions visant à uniformiser les procédures en vigueur%
}

\newgls{CONOPS}{Concept of Operation}{conops}{%
Document qui décrit de manière claire et concise les objectifs du JFC et ses intentions quant à la façon de les accomplir%
}

\newgls{COMPLAN}{Communications Plan}{complan}{%
Plan de communications (réseaux, canaux, fréquence, procédures)%
}

\newgls{SA}{Situationnal Awareness}{sa}{%
La perception de la situation est la capacité de se représenter l'environnement autour de soi, et à s'en fabriquer une image mentale persistante. Un pilote chevronné parviendra également à projeter cet environnement dans le temps, et à réagir au danger avant qu'il ne devienne un facteur%
}

\newgls{LGB}{Laser Guided Bomb}{lgb}{%
Bombe guidée par laser
}

\newgls{LGW}{Laser Guided Weapon}{lgw}{%
Arme guidée par laser
}

\newgls{LTD}{Laser Target Designator}{ltd}{%
Désignateur de cible laser
}

\newgls{LSS}{Laser Spot Search}{lss}{%
Recherche de point laser
}

\newgls{FAH}{Final Attack Heading}{fah}{%
Cap suivi par l'appareil qui attaque lorsqu'il effectue on attaque. Ce cap est pris au moment du virage "IN" et est maintenu jusqu'au largage de la munition
}

\newgls{LST}{Laser Spot Tracker}{lst}{%
Détecteur de point laser
}

\newgls{PGM}{Precision Guided Munition}{pgm}{%
Munition intelligente guidée
}

\newgls{NVG}{Night Vision Goggles}{nvg}{%
Lunettes de vision nocturne (infrarouge)
}

\newgls{ISR}{Intelligence, Surveillance, Reconnaissance}{isr}{%
Espionnage, surveillance et reconnaissance
}

\newgls{ELINT}{Electronic Intelligence}{elint}{%
Espionnage électronique
}

\newgls{IP}{Initial Point}{ip}{%
Point initial
}

\newgls{EW}{Electronic Warfare}{ew}{%
Guerre électronique
}

\newgls{EO}{Electro-Optics}{eo}{%
Électro-optique (=télévision)
}

\newgls{GRG}{Gridded Reference Graphic}{grg}{%
Graphique de référence quadrillé
}

\newgls{LOS}{Line Of Sight}{los}{%
Ligne de vue
}

\newgls{PLA}{Post Launch Annulation}{pla}{%
Annulation de l'attaque après que l'arme ait été tirée
}

\newgls{TLE}{Target Location Error}{tle}{%
Ligne de vue
}

\newacronym{misrep}{MISREP}{Mission Report}
\newacronym{cc}{C2}{Command and Control}
\newacronym{ccc}{C3}{Command, Control and Communications}
\newacronym{cas}{CAS}{Close Air Support}
\newacronym{awacs}{AWACS}{Airborne Warning And Control System}
\newacronym{rw}{RW}{Rotary-Wing}
\newacronym{fw}{FW}{Fixed-Wing}
\newacronym{farp}{FARP}{Forward Arming and Refueling Point}
\newacronym{tot}{TOT}{Time On Target}
\newacronym{ttt}{TTT}{Time To Target}
\newacronym{bp}{BP}{Battle Position}
\newacronym{ha}{HA}{Holding Area}
\newacronym{fp}{FP}{Firing Point}
\newacronym{tacon}{TACON}{TActical CONtrol}
\newacronym{noe}{NOE}{Nap Of the Earth}
\newacronym{rpg}{RPG}{Rocket Propelled Grenade}
\newacronym{ttp}{TTP}{Tactics, Techniques, and Procedures}
\newacronym{sam}{SAM}{Surface to Air Missile}
\newacronym{aaa}{AAA}{Anti Air Artillery}
\newacronym{ewr}{EWR}{Early Warning Radar}
\newacronym{recce}{RECCE}{Reconnaissance}
\newacronym{sparkle}{sparkle}{Marque infrarouge}
\newacronym{ir}{IR}{Infrarouge}
\newacronym{ks}{Killer Scout}{Sobriquet pour le FAC(A)}
\newacronym{abort}{ABORT}{Appel: "Cessez action/attaque/mission"}
\newacronym{clearedhot}{CLEARED HOT}{Appel (contrôle de types 1 et 2): "Autorisé à larguer de l'armement pour cette passe"}
\newacronym{continue}{CONTINUE}{Appel: "Poursuivez la manoeuvre en cours" (n'implique pas l'autorisation de tir)}
\newacronym{continuedry}{CONTINUE DRY}{Appel: "Poursuivez la manoeuvre en cours, tir interdit" (pour le contrôle de type 3, le TAC doit utiliser "Type 3, CONTINUE DRY") (utilisé pour l'entraînement ou le "Show of force")}
\newacronym{clearedeng}{CLEARED TO ENGAGE}{Appel (contrôle de Type 3): "L'appareil ou la formation est autorisée à commencer l'engagement (tirer) selon les paramètres établis avec le TAC"}
\newacronym{commencing}{COMMENCING ENGAGEMENT}{Appel (contrôle de Type 3) émis par le pilote pour signaler le début d'une attaque de Type 3}
\newacronym{complete}{CLEARED TO ENGAGE}{Appel (contrôle de Type 3) émis par le pilote pour signaler la fin d'un engagement de Type 3}
\newacronym{ipinbound}{IP INBOUND}{Appel émis par le pilote pour signaler qu'il se trouve sur l'IP}
\newacronym{in}{IN}{Appel émis par le pilote pour signaler qu'il est sur le point d'engager la cible et atteint l'autorisation ("CLEARED HOT")}

