\newcommand{\newgls}[4]{% 1: abbrev 2: full name 3: ref 4: desc
	\newglossaryentry{#3g}{%
		name={#1},%
		description={#4}%
	}%
	\newglossaryentry{#3}{%
		type=\acronymtype,%
		name={#1},%
		description={#2},%
		first={#2 (#1)\glsadd{#3g}},%
		see=[Explication:]{#3g}%
	}%
}

\newgls{SEAD}{Suppression of Ennemy Air Defence}{sead}{%
Aussi appelé ``Wild Weasel'', ou ``Iron Hand''. Action visant à neutraliser les défenses anti-aériennes ennemies, pas uniquement les SAMs ou AAA, mais également mes EWRs et le C3. Cette action peut être effectuée au moyen d'armement, ou de moyens de guerre électronique.
}

\newgls{CID}{Combat Identification}{cid}{%
Capacité de différencier les cibles potentielles, mobiles ou fixes, dans de grandes zones, comme amies, ennemies ou neutre, dans un temps très court, avec un grand degré de certitude, et à la distance suffisante, pour permettre les décisions d'emploi de la force
}

\newgls{AOD}{Air Operation Directive}{aod}{%
Directive émise par le JFACC concernant la conduite des opération aériennes%
}

\newgls{AI}{Air Interdiction}{ai}{%
Aussi appelé DAS, pour ``Deep Air Support''. L'interdiction aérienne consiste à frapper des cibles au sol ennemies qui ne sont pas à proximité de troupes alliées. L'AI diffère du bombardement de par son but, qui reste le soutien des troupes au sol alliées%
}

\newgls{ROE}{Rule of Engagement}{roe}{%
Décrit les circonstances, les conditions, la manière et la proportion dans lesquelles l'usage de la force est autorisé%
}

\newgls{TGO}{Terminal Guidance Operation}{tgo}{%
Action de fournir aux appareils ou aux armes des informations supplémentaires relatives à la cible par la voix, l'électronique, les communications visuelles ou des moyens mécaniques%
}

\newgls{TAC}{Terminal Attack Control}{tac}{%
Autorité qui contrôle la manœuvre %
}

\newgls{BDA}{Battle Damage Assessment}{bda}{%
Évaluation des dégâts infligés à la cible, ou aux cibles, suite à un engagement%
}

\newgls{FAC(A)}{Forward Air Controller}{faca}{%
Un pilote spécialement qualifié qui contrôle, depuis les airs, d'autres appareils engagés en opération de CAS%
}

\newgls{J-TAC}{Joint Terminal Attack Controller}{jtac}{%
Un membre du personnel qualifié qui contrôle, depuis une position au sol avancée, les appareils engagés en opération de CAS ou d'autres types d'opération offensive%
}

\newgls{GCAS}{Ground alert Close Air Support}{gcas}%
{CAS à la demande, avec les appareils en attente sur le parking}%

\newgls{XCAS}{Air alert Close Air Support}{xcas}%
{CAS à la demande, avec les appareils en attente en vol}

\newgls{GC}{Ground Commander}{gc}{Commandant des troupes au sol}%

\newgls{JFC}{Joint Force Commander}{jfc}{Commandant de la coalition alliée}%

\newgls{JFACC}{Joint Force Air Component Commander}{jfacc}{Commandant de la composante aérienne de la coalition alliée}%

\newgls{AOB}{Air Order of battle}{aob}{%
Escadrons, types d'appareils et base d'attache des appareils de la coalition alliée%
}

\newgls{ATO}{Air Tasking Order}{ato}{%
Document reprenant toutes les sorties aériennes prévues dans les 24 prochaines heures%
}

\newgls{ACO}{Airspace Control Order}{aco}{%
Document reprenant toutes les mesures de contrôle et de coordination de l'espace aérien en vigueur%
}

\newgls{SPINS}{Special Instructions}{spins}{%
Document reprenant toutes les procédures spécifiques à une opération, un escadron, une sortie, etc%
}

\newgls{OPORD}{Operations Order}{opord}{%
Plan visant à assister les unités subordonnées lors de l'exécution de leur tâche. Ce plan contient la situation à laquelle les unités doivent faire face, la mission de l'unité, ainsi que ce que l'on attend de l'unité. Cfr \url{https://en.wikipedia.org/wiki/Operations_order} pour un exemple d'OPORD%
}

\newgls{SOP}{Standard Operating Procedure}{sop}{%
Ensemble d'instructions visant à uniformiser les procédures en vigueur%
}

\newgls{CONOPS}{Concept of Operation}{conops}{%
Document qui décrit de manière claire et concise les objectifs du JFC et ses intentions quant à la façon de les accomplir%
}

\newgls{COMPLAN}{Communications Plan}{complan}{%
Plan de communications (réseaux, canaux, fréquence, procédures)%
}

\newgls{SA}{Situationnal Awareness}{sa}{%
La perception de la situation est la capacité de se représenter l'environnement autour de soi, et à s'en fabriquer une image mentale persistante. Un pilote chevronné parviendra également à projeter cet environnement dans le temps, et à réagir au danger avant qu'il ne devienne un facteur%
}

\newacronym{misrep}{MISREP}{Mission Report}
\newacronym{cc}{C2}{Command and Control}
\newacronym{ccc}{C3}{Command, Control and Communications}
\newacronym{cas}{CAS}{Close Air Support}
\newacronym{awacs}{AWACS}{Airborne Warning And Control System}
\newacronym{rw}{RW}{Rotary-Wing}
\newacronym{fw}{FW}{Fixed-Wing}
\newacronym{farp}{FARP}{Forward Arming and Refueling Point}
\newacronym{tot}{TOT}{Time On Target}
\newacronym{ttt}{TTT}{Time To Target}
\newacronym{bp}{BP}{Battle Position}
\newacronym{ha}{HA}{Holding Area}
\newacronym{fp}{FP}{Firing Point}
\newacronym{tacon}{TACON}{TActical CONtrol}
\newacronym{noe}{NOE}{Nap Of the Earth}
\newacronym{rpg}{RPG}{Rocket Propelled Grenade}
\newacronym{ttp}{TTP}{Tactics, Techniques, and Procedures}
\newacronym{sam}{SAM}{Surface to Air Missile}
\newacronym{aaa}{AAA}{Anti Air Artillery}
\newacronym{ewr}{EWR}{Early Warning Radar}
