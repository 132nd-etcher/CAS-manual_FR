\chapter{Carte de mission}\label[annex]{ann2}


%\renewcommand{\arraystretch}{6}
%\begin{table}[!htb]%
%\centering%\tiny%
%\begin{tabular}{ | p {0.2\textwidth} | p {0.2\textwidth} | p {0.2\textwidth} | p {0.2\textwidth} | }%
%\hline%
%\multicolumn{2}{ | p {0.4\textwidth} | }{\textbf{Mission}} & \multicolumn{2}{ p {0.4\textwidth} | }{\textbf{Date}} \\ \hline
%Call-sign  & T/O time & Freq pri. & Freq sec. \\ \hline
%                  &                  &                  &                  \\
%                  &                  &                  &                  \\
%                  &                  &                  &                  \\
%                  &                  &                  &                  \\
%                  &                  &                  &                  \\
%                  &                  &                  &                  \\
%                  &                  &                  &                  \\
%                  &                  &                  &                  \\
%                  &                  &                  &                  \\
%                  &                  &                  &                  \\ \hline
%\end{tabular}%


%\noindent
%\begin{tabularx}{\textwidth}{ | X | c | }
%  \hline
%  text & top\\
%  \hline
%  \noindent\parbox[c]{\hsize}{text} & center\\
%  \hline
%  \noindent\parbox[b]{\hsize}{text} & bottom\\
%  \hline
%\end{tabularx}

\noindent
    \begin{tabularx}{\textwidth}{lX}
        \toprule
        Mittel    & technische Details\\
        \midrule
        %first row of tabularx
        Laptop      &
            \begin{tabular}[t]{ll}
            MacBook Pro & \\
            Baujahr 2010 & \\ 
            Prozessor & 2.4 GHz Intel Core 2 Duo \\
            Speicher & 4 GB 1067 MHz DDR3 \\
            Software & Mac OS X Lion 10.7.5 (11G63)
            \end{tabular}\\[3em]
        %second row of tabularx
        Programmierumgebung          &
            \begin{tabular}[t]{ll}
            Version & 8.0.4.0 \\
            Plattform & Mac OS x86 (32-bit, 64-bit Kernel)
            \end{tabular}\\
        \bottomrule
    \end{tabularx}

\caption{My caption}
\label{my-label}
\end{table}
\renewcommand{\arraystretch}{1}