\chapter{Guide de préparation de mission}\label[annex]{ann1}

\e
    \item Situation alliée
    \ee
        \item FLOT
        \item Points remarquables
        \eee
            \item Points de contact
            \item Points de rapportages
            \item Points initiaux
        \ed
        \item Dispositif allié
        \eee
            \item Zone d’opération
            \item Type de terrain
            \item Position et callsigns du J-TAC/FAC(A)
            \item Unités alliées en support
        \ed
        \item Contrôle et coordination
        \eee
            \item Mesures permissives
            \item Mesures restrictives
        \ed
        \item Gestion de l’espace aérien
        \item Zones d’engagement de la chasse
    \ed
    \item Situation ennemie
    \ee
        \item Position et force de l’ennemi
        \eee
            \item Intention supposée
            \item Route de déplacement probable
            \item Tactiques déjà observées
        \ed
        \item Éléments de support ennemis
        \item Menaces
        \eee
            \item Position
            \item Type de guidage
            \eeee
                \item Infrarouge
                \item Radar
                \item Optique
            \ed
            \item Capacité de la menace (portée, puissance)
            \item Éléments révélateurs
            \item Tactiques déjà observées
        \ed
    \ed
    \item Météo
    \ee
        \item Plafond
        \item Visibilité
        \item Température
        \item Vent
    \ed
    \item Environnement
    \ee
        \item Élévation et azimuth du soleil
        \item Lever et coucher du soleil
    \ed
    \item Objectifs de mission
    \ee
        \item Objectifs globaux de l’opération
        \item Objectifs du haut commandement
        \item Objectifs de la sortie
        \item Objectifs des unités alliées
        \item Ordre de priorité des différentes cibles potentielles
        \item Heure d’arrivée prévue/imposée sur objectif
        \item Règles d’engagement
    \ed
\ed




